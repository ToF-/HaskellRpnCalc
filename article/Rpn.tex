\documentclass{article}

%% ODER: format ==         = "\mathrel{==}"
%% ODER: format /=         = "\neq "
%
%
\makeatletter
\@ifundefined{lhs2tex.lhs2tex.sty.read}%
  {\@namedef{lhs2tex.lhs2tex.sty.read}{}%
   \newcommand\SkipToFmtEnd{}%
   \newcommand\EndFmtInput{}%
   \long\def\SkipToFmtEnd#1\EndFmtInput{}%
  }\SkipToFmtEnd

\newcommand\ReadOnlyOnce[1]{\@ifundefined{#1}{\@namedef{#1}{}}\SkipToFmtEnd}
\usepackage{amstext}
\usepackage{amssymb}
\usepackage{stmaryrd}
\DeclareFontFamily{OT1}{cmtex}{}
\DeclareFontShape{OT1}{cmtex}{m}{n}
  {<5><6><7><8>cmtex8
   <9>cmtex9
   <10><10.95><12><14.4><17.28><20.74><24.88>cmtex10}{}
\DeclareFontShape{OT1}{cmtex}{m}{it}
  {<-> ssub * cmtt/m/it}{}
\newcommand{\texfamily}{\fontfamily{cmtex}\selectfont}
\DeclareFontShape{OT1}{cmtt}{bx}{n}
  {<5><6><7><8>cmtt8
   <9>cmbtt9
   <10><10.95><12><14.4><17.28><20.74><24.88>cmbtt10}{}
\DeclareFontShape{OT1}{cmtex}{bx}{n}
  {<-> ssub * cmtt/bx/n}{}
\newcommand{\tex}[1]{\text{\texfamily#1}}	% NEU

\newcommand{\Sp}{\hskip.33334em\relax}


\newcommand{\Conid}[1]{\mathit{#1}}
\newcommand{\Varid}[1]{\mathit{#1}}
\newcommand{\anonymous}{\kern0.06em \vbox{\hrule\@width.5em}}
\newcommand{\plus}{\mathbin{+\!\!\!+}}
\newcommand{\bind}{\mathbin{>\!\!\!>\mkern-6.7mu=}}
\newcommand{\rbind}{\mathbin{=\mkern-6.7mu<\!\!\!<}}% suggested by Neil Mitchell
\newcommand{\sequ}{\mathbin{>\!\!\!>}}
\renewcommand{\leq}{\leqslant}
\renewcommand{\geq}{\geqslant}
\usepackage{polytable}

%mathindent has to be defined
\@ifundefined{mathindent}%
  {\newdimen\mathindent\mathindent\leftmargini}%
  {}%

\def\resethooks{%
  \global\let\SaveRestoreHook\empty
  \global\let\ColumnHook\empty}
\newcommand*{\savecolumns}[1][default]%
  {\g@addto@macro\SaveRestoreHook{\savecolumns[#1]}}
\newcommand*{\restorecolumns}[1][default]%
  {\g@addto@macro\SaveRestoreHook{\restorecolumns[#1]}}
\newcommand*{\aligncolumn}[2]%
  {\g@addto@macro\ColumnHook{\column{#1}{#2}}}

\resethooks

\newcommand{\onelinecommentchars}{\quad-{}- }
\newcommand{\commentbeginchars}{\enskip\{-}
\newcommand{\commentendchars}{-\}\enskip}

\newcommand{\visiblecomments}{%
  \let\onelinecomment=\onelinecommentchars
  \let\commentbegin=\commentbeginchars
  \let\commentend=\commentendchars}

\newcommand{\invisiblecomments}{%
  \let\onelinecomment=\empty
  \let\commentbegin=\empty
  \let\commentend=\empty}

\visiblecomments

\newlength{\blanklineskip}
\setlength{\blanklineskip}{0.66084ex}

\newcommand{\hsindent}[1]{\quad}% default is fixed indentation
\let\hspre\empty
\let\hspost\empty
\newcommand{\NB}{\textbf{NB}}
\newcommand{\Todo}[1]{$\langle$\textbf{To do:}~#1$\rangle$}

\EndFmtInput
\makeatother
%
%



\begin{document}
\setlength{\parindent}{0em}
This is the module for the reverse polish notation calculator. 
\begingroup\par\noindent\advance\leftskip\mathindent\(
\begin{pboxed}\SaveRestoreHook
\column{B}{@{}>{\hspre}l<{\hspost}@{}}%
\column{E}{@{}>{\hspre}l<{\hspost}@{}}%
\>[B]{}\mathbf{module}\;\Conid{Rpn}{}\<[E]%
\\
\>[B]{}\mathbf{where}{}\<[E]%
\ColumnHook
\end{pboxed}
\)\par\noindent\endgroup\resethooks
The calculator is given a \text{\ttfamily String} containing an RPN expression, and yields the value of this expression, a \text{\ttfamily Number}. Here we choose \text{\ttfamily Number} to be equivalent to \text{\ttfamily Integer}, but another numeric type would also work.
\begingroup\par\noindent\advance\leftskip\mathindent\(
\begin{pboxed}\SaveRestoreHook
\column{B}{@{}>{\hspre}l<{\hspost}@{}}%
\column{E}{@{}>{\hspre}l<{\hspost}@{}}%
\>[B]{}\mathbf{type}\;\Conid{Number}\mathrel{=}\Conid{Integer}{}\<[E]%
\\
\>[B]{}\Varid{rpn}\mathbin{::}\Conid{String}\to \Conid{Number}{}\<[E]%
\\
\>[B]{}\Varid{rpn}\mathrel{=}\bot {}\<[E]%
\ColumnHook
\end{pboxed}
\)\par\noindent\endgroup\resethooks
The meaning of an RPN expression can be expressed as a list of \emph{tokens}. A token is either a value or an operator. Most operators operate on two numbers like \text{\ttfamily \char43{}} or \text{\ttfamily \char42{}}:they are called \emph{binary} operators, and some of them are \emph{unary} operators, like \text{\ttfamily \char33{}}. Let's describe this in Haskell:
\begingroup\par\noindent\advance\leftskip\mathindent\(
\begin{pboxed}\SaveRestoreHook
\column{B}{@{}>{\hspre}l<{\hspost}@{}}%
\column{12}{@{}>{\hspre}l<{\hspost}@{}}%
\column{21}{@{}>{\hspre}l<{\hspost}@{}}%
\column{E}{@{}>{\hspre}l<{\hspost}@{}}%
\>[B]{}\mathbf{data}\;\Conid{Token}\mathrel{=}\Conid{Value}\;\Conid{Number}{}\<[E]%
\\
\>[B]{}\hsindent{12}{}\<[12]%
\>[12]{}\mid \Conid{Unary}\;{}\<[21]%
\>[21]{}(\Conid{Number}\to \Conid{Number}){}\<[E]%
\\
\>[B]{}\hsindent{12}{}\<[12]%
\>[12]{}\mid \Conid{Binary}\;(\Conid{Number}\to \Conid{Number}\to \Conid{Number}){}\<[E]%
\ColumnHook
\end{pboxed}
\)\par\noindent\endgroup\resethooks
To evaluate the expression, we execute each token on a \emph{stack}, a structure that allows for inserting and removing numbers in a \emph{last in, first out} fashion. A stack can either be empty (in which case applying an operator should result in a program error) or have an item followed by a possibly empty sequence of item previously entered on the stack.
\begingroup\par\noindent\advance\leftskip\mathindent\(
\begin{pboxed}\SaveRestoreHook
\column{B}{@{}>{\hspre}l<{\hspost}@{}}%
\column{5}{@{}>{\hspre}l<{\hspost}@{}}%
\column{12}{@{}>{\hspre}l<{\hspost}@{}}%
\column{E}{@{}>{\hspre}l<{\hspost}@{}}%
\>[B]{}\mathbf{data}\;\Conid{Stack}\mathrel{=}\Conid{Empty}{}\<[E]%
\\
\>[B]{}\hsindent{12}{}\<[12]%
\>[12]{}\mid \Conid{Item}\;\Conid{Number}\;\Conid{Stack}{}\<[E]%
\\
\>[B]{}\hsindent{5}{}\<[5]%
\>[5]{}\mathbf{deriving}\;(\Conid{Show}){}\<[E]%
\ColumnHook
\end{pboxed}
\)\par\noindent\endgroup\resethooks
Two functions are defined that use a \text{\ttfamily Stack}. The first one, \text{\ttfamily push}, takes an existing stack, a number, and puts that number on top of the existing stack, thus yielding a new stack.
\begingroup\par\noindent\advance\leftskip\mathindent\(
\begin{pboxed}\SaveRestoreHook
\column{B}{@{}>{\hspre}l<{\hspost}@{}}%
\column{E}{@{}>{\hspre}l<{\hspost}@{}}%
\>[B]{}\Varid{push}\mathbin{::}\Conid{Stack}\to \Conid{Number}\to \Conid{Stack}{}\<[E]%
\\
\>[B]{}\Varid{push}\;\Varid{st}\;\Varid{n}\mathrel{=}\Conid{Item}\;\Varid{n}\;\Varid{st}{}\<[E]%
\ColumnHook
\end{pboxed}
\)\par\noindent\endgroup\resethooks
The second one, \text{\ttfamily pop}, extracts the number at the top and returns this number and the stack without this number. Popping a number from the empty stack results in a progam error. 
\begingroup\par\noindent\advance\leftskip\mathindent\(
\begin{pboxed}\SaveRestoreHook
\column{B}{@{}>{\hspre}l<{\hspost}@{}}%
\column{E}{@{}>{\hspre}l<{\hspost}@{}}%
\>[B]{}\Varid{pop}\mathbin{::}\Conid{Stack}\to (\Conid{Number},\Conid{Stack}){}\<[E]%
\\
\>[B]{}\Varid{pop}\;(\Conid{Item}\;\Varid{n}\;\Varid{st})\mathrel{=}(\Varid{n},\Varid{st}){}\<[E]%
\\
\>[B]{}\Varid{pop}\;\Conid{Empty}\mathrel{=}\Varid{error}\;\text{\ttfamily \char34 empty~stack\char34}{}\<[E]%
\ColumnHook
\end{pboxed}
\)\par\noindent\endgroup\resethooks
(Here's some function to present the content of a stack as a list of numbers and vice versa)
\begingroup\par\noindent\advance\leftskip\mathindent\(
\begin{pboxed}\SaveRestoreHook
\column{B}{@{}>{\hspre}l<{\hspost}@{}}%
\column{5}{@{}>{\hspre}l<{\hspost}@{}}%
\column{11}{@{}>{\hspre}l<{\hspost}@{}}%
\column{E}{@{}>{\hspre}l<{\hspost}@{}}%
\>[B]{}\Varid{toList}\mathbin{::}\Conid{Stack}\to [\mskip1.5mu \Conid{Number}\mskip1.5mu]{}\<[E]%
\\
\>[B]{}\Varid{toList}\mathrel{=}\Varid{reverse}\mathbin{\circ}\Varid{toList'}{}\<[E]%
\\
\>[B]{}\hsindent{5}{}\<[5]%
\>[5]{}\mathbf{where}{}\<[E]%
\\
\>[B]{}\hsindent{5}{}\<[5]%
\>[5]{}\Varid{toList'}\;\Conid{Empty}\mathrel{=}[\mskip1.5mu \mskip1.5mu]{}\<[E]%
\\
\>[B]{}\hsindent{5}{}\<[5]%
\>[5]{}\Varid{toList'}\;\Varid{st}\mathrel{=}\Varid{n}\mathbin{:}\Varid{toList'}\;\Varid{st'}\;\mathbf{where}\;(\Varid{n},\Varid{st'})\mathrel{=}\Varid{pop}\;\Varid{st}{}\<[E]%
\\[\blanklineskip]%
\>[B]{}\Varid{fromList}\mathbin{::}[\mskip1.5mu \Conid{Number}\mskip1.5mu]\to \Conid{Stack}{}\<[E]%
\\
\>[B]{}\Varid{fromList}{}\<[11]%
\>[11]{}\mathrel{=}\Varid{foldl}\;\Varid{push}\;\Conid{Empty}{}\<[E]%
\ColumnHook
\end{pboxed}
\)\par\noindent\endgroup\resethooks
Applying the token \text{\ttfamily Number~} $n$ on the stack results in pushing the integer $n$ on the top op the stack. Applying an unary operator consists in applying the operator on the top of the stack, while applying a binary operator is done by pulling two values from the stack, then pushing the result of the operation. 
\begingroup\par\noindent\advance\leftskip\mathindent\(
\begin{pboxed}\SaveRestoreHook
\column{B}{@{}>{\hspre}l<{\hspost}@{}}%
\column{5}{@{}>{\hspre}l<{\hspost}@{}}%
\column{9}{@{}>{\hspre}l<{\hspost}@{}}%
\column{39}{@{}>{\hspre}l<{\hspost}@{}}%
\column{E}{@{}>{\hspre}l<{\hspost}@{}}%
\>[B]{}\Varid{apply}\mathbin{::}\Conid{Stack}\to \Conid{Token}\to \Conid{Stack}{}\<[E]%
\\
\>[B]{}\Varid{apply}\;\Varid{st}\;(\Conid{Value}\;\Varid{n})\mathrel{=}\Conid{Item}\;\Varid{n}\;\Varid{st}{}\<[E]%
\\
\>[B]{}\Varid{apply}\;\Varid{st}\;(\Conid{Unary}\;\Varid{op})\mathrel{=}\mathbf{let}\;(\Varid{n},\Varid{st'}){}\<[39]%
\>[39]{}\mathrel{=}\Varid{pop}\;\Varid{st}\;\mathbf{in}\;\Conid{Item}\;(\Varid{op}\;\Varid{n})\;\Varid{st'}{}\<[E]%
\\
\>[B]{}\Varid{apply}\;\Varid{st}\;(\Conid{Binary}\;\Varid{op})\mathrel{=}{}\<[E]%
\\
\>[B]{}\hsindent{5}{}\<[5]%
\>[5]{}\mathbf{let}\;(\Varid{n},\Varid{st'})\mathrel{=}\Varid{pop}\;\Varid{st}{}\<[E]%
\\
\>[5]{}\hsindent{4}{}\<[9]%
\>[9]{}(\Varid{n'},\Varid{st''})\mathrel{=}\Varid{pop}\;\Varid{st'}{}\<[E]%
\\
\>[B]{}\hsindent{5}{}\<[5]%
\>[5]{}\mathbf{in}\;\Conid{Item}\;(\Varid{n'}\mathbin{`\Varid{op}`}\Varid{n})\;\Varid{st''}{}\<[E]%
\ColumnHook
\end{pboxed}
\)\par\noindent\endgroup\resethooks
We can now calculte values expressed as a list of RPN tokens: for example the following list will convert a temperature from Celsius to Fahreinheit:
\begingroup\par\noindent\advance\leftskip\mathindent\(
\begin{pboxed}\SaveRestoreHook
\column{B}{@{}>{\hspre}l<{\hspost}@{}}%
\column{5}{@{}>{\hspre}l<{\hspost}@{}}%
\column{E}{@{}>{\hspre}l<{\hspost}@{}}%
\>[B]{}\Varid{cToF}\mathbin{::}\Conid{Number}\to \Conid{Number}{}\<[E]%
\\
\>[B]{}\Varid{cToF}\;\Varid{n}\mathrel{=}\Varid{head}\mathbin{\$}\Varid{toList}\mathbin{\$}\Varid{foldl}\;\Varid{apply}\;(\Varid{fromList}\;[\mskip1.5mu \Varid{n}\mskip1.5mu]){}\<[E]%
\\
\>[B]{}\hsindent{5}{}\<[5]%
\>[5]{}[\mskip1.5mu \Conid{Value}\;\mathrm{9}{}\<[E]%
\\
\>[B]{}\hsindent{5}{}\<[5]%
\>[5]{},\Conid{Binary}\;(\mathbin{*}){}\<[E]%
\\
\>[B]{}\hsindent{5}{}\<[5]%
\>[5]{},\Conid{Value}\;\mathrm{5}{}\<[E]%
\\
\>[B]{}\hsindent{5}{}\<[5]%
\>[5]{},\Conid{Binary}\;\Varid{div}{}\<[E]%
\\
\>[B]{}\hsindent{5}{}\<[5]%
\>[5]{},\Conid{Value}\;\mathrm{32}{}\<[E]%
\\
\>[B]{}\hsindent{5}{}\<[5]%
\>[5]{},\Conid{Binary}\;(\mathbin{+})\mskip1.5mu]{}\<[E]%
\ColumnHook
\end{pboxed}
\)\par\noindent\endgroup\resethooks
THe remaining problem is to parse a string into a list of tokens.
\end{document}
