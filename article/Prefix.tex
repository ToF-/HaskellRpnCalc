\documentclass[a4paper,10pt]{article}
\usepackage{longtable,geometry}
\usepackage[english]{babel}
\usepackage[latin1]{inputenc}
\usepackage[babel]{csquotes}
\usepackage{multicol}
\usepackage{enumitem}
\usepackage{array}
\usepackage{fancyhdr}
\pagestyle{plain}
\geometry{dvips,a4paper,margin=1.5in}
\usepackage[usenames,dvipsnames]{xcolor}
\usepackage{amssymb,amsmath} 
\usepackage{afterpage}
\usepackage{pagecolor}
\usepackage{fancybox}
\usepackage{listings}
\usepackage{verbatim}
\usepackage{fancyvrb}
\usepackage{moreverb}
\usepackage{txfonts}
\usepackage{hyperref}
\usepackage{bytefield}
\usepackage{rotating}
\usepackage{tikz}
\usepackage{wallpaper}

%% ODER: format ==         = "\mathrel{==}"
%% ODER: format /=         = "\neq "
%
%
\makeatletter
\@ifundefined{lhs2tex.lhs2tex.sty.read}%
  {\@namedef{lhs2tex.lhs2tex.sty.read}{}%
   \newcommand\SkipToFmtEnd{}%
   \newcommand\EndFmtInput{}%
   \long\def\SkipToFmtEnd#1\EndFmtInput{}%
  }\SkipToFmtEnd

\newcommand\ReadOnlyOnce[1]{\@ifundefined{#1}{\@namedef{#1}{}}\SkipToFmtEnd}
\usepackage{amstext}
\usepackage{amssymb}
\usepackage{stmaryrd}
\DeclareFontFamily{OT1}{cmtex}{}
\DeclareFontShape{OT1}{cmtex}{m}{n}
  {<5><6><7><8>cmtex8
   <9>cmtex9
   <10><10.95><12><14.4><17.28><20.74><24.88>cmtex10}{}
\DeclareFontShape{OT1}{cmtex}{m}{it}
  {<-> ssub * cmtt/m/it}{}
\newcommand{\texfamily}{\fontfamily{cmtex}\selectfont}
\DeclareFontShape{OT1}{cmtt}{bx}{n}
  {<5><6><7><8>cmtt8
   <9>cmbtt9
   <10><10.95><12><14.4><17.28><20.74><24.88>cmbtt10}{}
\DeclareFontShape{OT1}{cmtex}{bx}{n}
  {<-> ssub * cmtt/bx/n}{}
\newcommand{\tex}[1]{\text{\texfamily#1}}	% NEU

\newcommand{\Sp}{\hskip.33334em\relax}


\newcommand{\Conid}[1]{\mathit{#1}}
\newcommand{\Varid}[1]{\mathit{#1}}
\newcommand{\anonymous}{\kern0.06em \vbox{\hrule\@width.5em}}
\newcommand{\plus}{\mathbin{+\!\!\!+}}
\newcommand{\bind}{\mathbin{>\!\!\!>\mkern-6.7mu=}}
\newcommand{\rbind}{\mathbin{=\mkern-6.7mu<\!\!\!<}}% suggested by Neil Mitchell
\newcommand{\sequ}{\mathbin{>\!\!\!>}}
\renewcommand{\leq}{\leqslant}
\renewcommand{\geq}{\geqslant}
\usepackage{polytable}

%mathindent has to be defined
\@ifundefined{mathindent}%
  {\newdimen\mathindent\mathindent\leftmargini}%
  {}%

\def\resethooks{%
  \global\let\SaveRestoreHook\empty
  \global\let\ColumnHook\empty}
\newcommand*{\savecolumns}[1][default]%
  {\g@addto@macro\SaveRestoreHook{\savecolumns[#1]}}
\newcommand*{\restorecolumns}[1][default]%
  {\g@addto@macro\SaveRestoreHook{\restorecolumns[#1]}}
\newcommand*{\aligncolumn}[2]%
  {\g@addto@macro\ColumnHook{\column{#1}{#2}}}

\resethooks

\newcommand{\onelinecommentchars}{\quad-{}- }
\newcommand{\commentbeginchars}{\enskip\{-}
\newcommand{\commentendchars}{-\}\enskip}

\newcommand{\visiblecomments}{%
  \let\onelinecomment=\onelinecommentchars
  \let\commentbegin=\commentbeginchars
  \let\commentend=\commentendchars}

\newcommand{\invisiblecomments}{%
  \let\onelinecomment=\empty
  \let\commentbegin=\empty
  \let\commentend=\empty}

\visiblecomments

\newlength{\blanklineskip}
\setlength{\blanklineskip}{0.66084ex}

\newcommand{\hsindent}[1]{\quad}% default is fixed indentation
\let\hspre\empty
\let\hspost\empty
\newcommand{\NB}{\textbf{NB}}
\newcommand{\Todo}[1]{$\langle$\textbf{To do:}~#1$\rangle$}

\EndFmtInput
\makeatother
%
%


%
%
%
%
%
% This package provides two environments suitable to take the place
% of hscode, called "plainhscode" and "arrayhscode". 
%
% The plain environment surrounds each code block by vertical space,
% and it uses \abovedisplayskip and \belowdisplayskip to get spacing
% similar to formulas. Note that if these dimensions are changed,
% the spacing around displayed math formulas changes as well.
% All code is indented using \leftskip.
%
% Changed 19.08.2004 to reflect changes in colorcode. Should work with
% CodeGroup.sty.
%
\ReadOnlyOnce{polycode.fmt}%
\makeatletter

\newcommand{\hsnewpar}[1]%
  {{\parskip=0pt\parindent=0pt\par\vskip #1\noindent}}

% can be used, for instance, to redefine the code size, by setting the
% command to \small or something alike
\newcommand{\hscodestyle}{}

% The command \sethscode can be used to switch the code formatting
% behaviour by mapping the hscode environment in the subst directive
% to a new LaTeX environment.

\newcommand{\sethscode}[1]%
  {\expandafter\let\expandafter\hscode\csname #1\endcsname
   \expandafter\let\expandafter\endhscode\csname end#1\endcsname}

% "compatibility" mode restores the non-polycode.fmt layout.

\newenvironment{compathscode}%
  {\par\noindent
   \advance\leftskip\mathindent
   \hscodestyle
   \let\\=\@normalcr
   \let\hspre\(\let\hspost\)%
   \pboxed}%
  {\endpboxed\)%
   \par\noindent
   \ignorespacesafterend}

\newcommand{\compaths}{\sethscode{compathscode}}

% "plain" mode is the proposed default.
% It should now work with \centering.
% This required some changes. The old version
% is still available for reference as oldplainhscode.

\newenvironment{plainhscode}%
  {\hsnewpar\abovedisplayskip
   \advance\leftskip\mathindent
   \hscodestyle
   \let\hspre\(\let\hspost\)%
   \pboxed}%
  {\endpboxed%
   \hsnewpar\belowdisplayskip
   \ignorespacesafterend}

\newenvironment{oldplainhscode}%
  {\hsnewpar\abovedisplayskip
   \advance\leftskip\mathindent
   \hscodestyle
   \let\\=\@normalcr
   \(\pboxed}%
  {\endpboxed\)%
   \hsnewpar\belowdisplayskip
   \ignorespacesafterend}

% Here, we make plainhscode the default environment.

\newcommand{\plainhs}{\sethscode{plainhscode}}
\newcommand{\oldplainhs}{\sethscode{oldplainhscode}}
\plainhs

% The arrayhscode is like plain, but makes use of polytable's
% parray environment which disallows page breaks in code blocks.

\newenvironment{arrayhscode}%
  {\hsnewpar\abovedisplayskip
   \advance\leftskip\mathindent
   \hscodestyle
   \let\\=\@normalcr
   \(\parray}%
  {\endparray\)%
   \hsnewpar\belowdisplayskip
   \ignorespacesafterend}

\newcommand{\arrayhs}{\sethscode{arrayhscode}}

% The mathhscode environment also makes use of polytable's parray 
% environment. It is supposed to be used only inside math mode 
% (I used it to typeset the type rules in my thesis).

\newenvironment{mathhscode}%
  {\parray}{\endparray}

\newcommand{\mathhs}{\sethscode{mathhscode}}

% texths is similar to mathhs, but works in text mode.

\newenvironment{texthscode}%
  {\(\parray}{\endparray\)}

\newcommand{\texths}{\sethscode{texthscode}}

% The framed environment places code in a framed box.

\def\codeframewidth{\arrayrulewidth}
\RequirePackage{calc}

\newenvironment{framedhscode}%
  {\parskip=\abovedisplayskip\par\noindent
   \hscodestyle
   \arrayrulewidth=\codeframewidth
   \tabular{@{}|p{\linewidth-2\arraycolsep-2\arrayrulewidth-2pt}|@{}}%
   \hline\framedhslinecorrect\\{-1.5ex}%
   \let\endoflinesave=\\
   \let\\=\@normalcr
   \(\pboxed}%
  {\endpboxed\)%
   \framedhslinecorrect\endoflinesave{.5ex}\hline
   \endtabular
   \parskip=\belowdisplayskip\par\noindent
   \ignorespacesafterend}

\newcommand{\framedhslinecorrect}[2]%
  {#1[#2]}

\newcommand{\framedhs}{\sethscode{framedhscode}}

% The inlinehscode environment is an experimental environment
% that can be used to typeset displayed code inline.

\newenvironment{inlinehscode}%
  {\(\def\column##1##2{}%
   \let\>\undefined\let\<\undefined\let\\\undefined
   \newcommand\>[1][]{}\newcommand\<[1][]{}\newcommand\\[1][]{}%
   \def\fromto##1##2##3{##3}%
   \def\nextline{}}{\) }%

\newcommand{\inlinehs}{\sethscode{inlinehscode}}

% The joincode environment is a separate environment that
% can be used to surround and thereby connect multiple code
% blocks.

\newenvironment{joincode}%
  {\let\orighscode=\hscode
   \let\origendhscode=\endhscode
   \def\endhscode{\def\hscode{\endgroup\def\@currenvir{hscode}\\}\begingroup}
   %\let\SaveRestoreHook=\empty
   %\let\ColumnHook=\empty
   %\let\resethooks=\empty
   \orighscode\def\hscode{\endgroup\def\@currenvir{hscode}}}%
  {\origendhscode
   \global\let\hscode=\orighscode
   \global\let\endhscode=\origendhscode}%

\makeatother
\EndFmtInput
%

\begin{document}
\setlength{\parindent}{0em}
Let's write a parser for expression in prefix notation. We will need built-in functions to detect a space or a digit character. 
\begin{hscode}\SaveRestoreHook
\column{B}{@{}>{\hspre}l<{\hspost}@{}}%
\column{E}{@{}>{\hspre}l<{\hspost}@{}}%
\>[B]{}\mathbf{module}\;\Conid{Prefix}{}\<[E]%
\\
\>[B]{}\mathbf{where}{}\<[E]%
\\
\>[B]{}\mathbf{import}\;\Conid{\Conid{Data}.Char}\;(\Varid{isSpace},\Varid{isDigit},\Varid{digitToInt}){}\<[E]%
\ColumnHook
\end{hscode}\resethooks
First let's define some operators that our parser will recognize.
\begin{hscode}\SaveRestoreHook
\column{B}{@{}>{\hspre}l<{\hspost}@{}}%
\column{8}{@{}>{\hspre}l<{\hspost}@{}}%
\column{E}{@{}>{\hspre}l<{\hspost}@{}}%
\>[B]{}\Varid{sPlus}\mathrel{=}\text{\ttfamily '+'}{}\<[E]%
\\
\>[B]{}\Varid{sMinus}\mathrel{=}\text{\ttfamily '-'}{}\<[E]%
\\
\>[B]{}\Varid{sMult}{}\<[8]%
\>[8]{}\mathrel{=}\text{\ttfamily '*'}{}\<[E]%
\\
\>[B]{}\Varid{sDiv}{}\<[8]%
\>[8]{}\mathrel{=}\text{\ttfamily '/'}{}\<[E]%
\\
\>[B]{}\Varid{sMod}{}\<[8]%
\>[8]{}\mathrel{=}\text{\ttfamily '\%'}{}\<[E]%
\\
\>[B]{}\Varid{sNegate}\mathrel{=}\text{\ttfamily '\char126 '}{}\<[E]%
\\
\>[B]{}\Varid{sFact}{}\<[8]%
\>[8]{}\mathrel{=}\text{\ttfamily '!'}{}\<[E]%
\ColumnHook
\end{hscode}\resethooks
A \emph{parser} is a function that breaks a string into components called \emph{tokens}.
\begin{hscode}\SaveRestoreHook
\column{B}{@{}>{\hspre}l<{\hspost}@{}}%
\column{E}{@{}>{\hspre}l<{\hspost}@{}}%
\>[B]{}\mathbf{type}\;\Conid{Parser}\mathrel{=}\Conid{String}\to [\mskip1.5mu (\Conid{Token},\Conid{String})\mskip1.5mu]{}\<[E]%
\ColumnHook
\end{hscode}\resethooks
Typically, a function of type \text{\ttfamily Parser} will examine its given String argument looking for a specific token. If the token is found, it will be returned in a list, along with the part of the string that remains to be parsed. If the token is not present, an empty list is returned. 
The token that can be parsed in a prefix expression is one of:
\begin{itemize}
\item A number,
\item An operator representing an unary function, followed by its operand,
\item A operator representing a binary function, followed by its two parameters. 
\end{itemize}
Each operand can be in turn, a full prefix expression. 
For instance, the expression \text{\ttfamily \char42{}\char43{}~42~17~\char33{}5} can be parsed into the \text{\ttfamily PrefExp} value: \\
\text{\ttfamily ~Op2~\char40{}\char42{}\char41{}}\\
\text{\ttfamily ~~~\char40{}Op2~\char40{}\char43{}\char41{}}\\
\text{\ttfamily ~~~~~~\char40{}Num~42\char41{}} \\
\text{\ttfamily ~~~~~~\char40{}Num~17\char41{}\char41{}}\\
\text{\ttfamily ~~~\char40{}Op1} (!)\\
\text{\ttfamily ~~~~~~\char40{}Num~5\char41{}\char41{}}\\

Since functions cannot be shown (try to enter \text{\ttfamily \char40{}\char43{}\char41{}} at the \emph{ghci} prompt to see why), we put additional information in the definition of \text{\ttfamily PrefExp} values so that showing them make sense.
\begin{hscode}\SaveRestoreHook
\column{B}{@{}>{\hspre}l<{\hspost}@{}}%
\column{14}{@{}>{\hspre}l<{\hspost}@{}}%
\column{E}{@{}>{\hspre}l<{\hspost}@{}}%
\>[B]{}\mathbf{type}\;\Conid{Number}\mathrel{=}\Conid{Integer}{}\<[E]%
\\
\>[B]{}\mathbf{type}\;\Conid{Token}\mathrel{=}\Conid{PrefExp}{}\<[E]%
\\
\>[B]{}\mathbf{type}\;\Conid{Symbol}\mathrel{=}\Conid{Char}{}\<[E]%
\\
\>[B]{}\mathbf{data}\;\Conid{PrefExp}\mathrel{=}\Conid{Num}\;\Conid{Number}{}\<[E]%
\\
\>[B]{}\hsindent{14}{}\<[14]%
\>[14]{}\mid \Conid{Op1}\;\Conid{Symbol}\;(\Conid{Number}\to \Conid{Number})\;\Conid{PrefExp}{}\<[E]%
\\
\>[B]{}\hsindent{14}{}\<[14]%
\>[14]{}\mid \Conid{Op2}\;\Conid{Symbol}\;(\Conid{Number}\to \Conid{Number}\to \Conid{Number})\;\Conid{PrefExp}\;\Conid{PrefExp}{}\<[E]%
\ColumnHook
\end{hscode}\resethooks
To make values of the type \text{\ttfamily PrefExp} visible (in ghci for example), we define its show function:
\begin{hscode}\SaveRestoreHook
\column{B}{@{}>{\hspre}l<{\hspost}@{}}%
\column{5}{@{}>{\hspre}l<{\hspost}@{}}%
\column{9}{@{}>{\hspre}l<{\hspost}@{}}%
\column{E}{@{}>{\hspre}l<{\hspost}@{}}%
\>[B]{}\mathbf{instance}\;\Conid{Show}\;\Conid{PrefExp}\;\mathbf{where}{}\<[E]%
\\
\>[B]{}\hsindent{5}{}\<[5]%
\>[5]{}\Varid{show}\;(\Conid{Num}\;\Varid{n})\mathrel{=}\Varid{show}\;\Varid{n}{}\<[E]%
\\
\>[B]{}\hsindent{5}{}\<[5]%
\>[5]{}\Varid{show}\;(\Conid{Op1}\;\Varid{c}\;\anonymous \;\Varid{prefExp})\mathrel{=}(\Varid{c}\mathbin{:}\text{\ttfamily \char34 ~\char34})\plus \Varid{show}\;\Varid{prefExp}{}\<[E]%
\\
\>[B]{}\hsindent{5}{}\<[5]%
\>[5]{}\Varid{show}\;(\Conid{Op2}\;\Varid{c}\;\anonymous \;\Varid{prefExp1}\;\Varid{prefExp2})\mathrel{=}{}\<[E]%
\\
\>[5]{}\hsindent{4}{}\<[9]%
\>[9]{}(\Varid{c}\mathbin{:}\text{\ttfamily \char34 ~\char34})\plus \Varid{show}\;\Varid{prefExp1}\plus \text{\ttfamily \char34 ~\char34}\plus \Varid{show}\;\Varid{prefExp2}{}\<[E]%
\ColumnHook
\end{hscode}\resethooks

Our first parser should recognize a digit, convert that value from \text{\ttfamily Int} and return the \text{\ttfamily Num} token.
\begin{hscode}\SaveRestoreHook
\column{B}{@{}>{\hspre}l<{\hspost}@{}}%
\column{5}{@{}>{\hspre}l<{\hspost}@{}}%
\column{E}{@{}>{\hspre}l<{\hspost}@{}}%
\>[B]{}\Varid{digit}\mathbin{::}\Conid{Parser}{}\<[E]%
\\
\>[B]{}\Varid{digit}\;(\Varid{c}\mathbin{:}\Varid{s})\mid \Varid{isDigit}\;\Varid{c}\mathrel{=}[\mskip1.5mu (\Varid{digitToNum}\;\Varid{c},\Varid{s})\mskip1.5mu]{}\<[E]%
\\
\>[B]{}\hsindent{5}{}\<[5]%
\>[5]{}\mathbf{where}\;\Varid{digitToNum}\mathrel{=}\Conid{Num}\mathbin{\circ}\Varid{fromIntegral}\mathbin{\circ}\Varid{digitToInt}{}\<[E]%
\\
\>[B]{}\Varid{digit}\;\anonymous \mathrel{=}[\mskip1.5mu \mskip1.5mu]{}\<[E]%
\ColumnHook
\end{hscode}\resethooks
Another parser converts all successive digits into a \text{\ttfamily Num} value.\\
\begin{tabular}{rl}
$0$ & 4807 \\
$0\times{10}+4$               & 807 \\
$(0\times{10}+4)\times{10}+8$ & 07 \\
$((0\times{10}+4)\times{10}+8)$ & 07 \\
$(((0\times{10}+4)\times{10}+8)\times{10}+0)$ & 7 \\
$((((0\times{10}+4)\times{10}+8)\times{10}+0)\times{10}+7)$ &  \\
\end{tabular}

\begin{hscode}\SaveRestoreHook
\column{B}{@{}>{\hspre}l<{\hspost}@{}}%
\column{5}{@{}>{\hspre}l<{\hspost}@{}}%
\column{E}{@{}>{\hspre}l<{\hspost}@{}}%
\>[B]{}\Varid{accum}\mathbin{::}\Conid{Integer}\to \Conid{Parser}{}\<[E]%
\\
\>[B]{}\Varid{accum}\;\Varid{acc}\;\Varid{s}\mathrel{=}\mathbf{case}\;\Varid{digit}\;\Varid{s}\;\mathbf{of}{}\<[E]%
\\
\>[B]{}\hsindent{5}{}\<[5]%
\>[5]{}[\mskip1.5mu \mskip1.5mu]\to [\mskip1.5mu (\Conid{Num}\;\Varid{acc},\Varid{s})\mskip1.5mu]{}\<[E]%
\\
\>[B]{}\hsindent{5}{}\<[5]%
\>[5]{}[\mskip1.5mu (\Conid{Num}\;\Varid{d},\Varid{s'})\mskip1.5mu]\to \Varid{accum}\;(\Varid{acc}\mathbin{*}\mathrm{10}\mathbin{+}\Varid{d})\;\Varid{s'}{}\<[E]%
\ColumnHook
\end{hscode}\resethooks
To parse a number, ignore spaces, then if one digit is found, accumulate all the following digits into a number. Otherwise if no digit was found, yield the empty result.
\begin{hscode}\SaveRestoreHook
\column{B}{@{}>{\hspre}l<{\hspost}@{}}%
\column{E}{@{}>{\hspre}l<{\hspost}@{}}%
\>[B]{}\Varid{number}\mathbin{::}\Conid{Parser}{}\<[E]%
\\
\>[B]{}\Varid{number}\;(\Varid{c}\mathbin{:}\Varid{s})\mid \Varid{isSpace}\;\Varid{c}\mathrel{=}\Varid{number}\;\Varid{s}{}\<[E]%
\\
\>[B]{}\Varid{number}\;(\Varid{c}\mathbin{:}\Varid{s})\mid \Varid{isDigit}\;\Varid{c}\mathrel{=}\Varid{accum}\;\mathrm{0}\;(\Varid{c}\mathbin{:}\Varid{s}){}\<[E]%
\\
\>[B]{}\Varid{number}\;\anonymous \mathrel{=}[\mskip1.5mu \mskip1.5mu]{}\<[E]%
\ColumnHook
\end{hscode}\resethooks
A parser for negation should recognize the symbol \text{\ttfamily \char96{}\char126{}\char96{}} and yield an unary operator token with the matching function. The same should ber done for the symbol \text{\ttfamily \char39{}\char33{}\char39{}} and the factorial operation. This can be generalized into a parser for any unary operator.
\begin{hscode}\SaveRestoreHook
\column{B}{@{}>{\hspre}l<{\hspost}@{}}%
\column{E}{@{}>{\hspre}l<{\hspost}@{}}%
\>[B]{}\Varid{unaryOp}\mathbin{::}\Conid{Symbole}\to (\Conid{Number}\to \Conid{Number})\to \Conid{Parser}{}\<[E]%
\\
\>[B]{}\Varid{unaryOp}\;\Varid{op}\;\Varid{f}\;(\Varid{c}\mathbin{:}\Varid{s})\mid \Varid{c}\equiv \Varid{op}\mathrel{=}[\mskip1.5mu (\Conid{Op1}\;\Varid{op}\;\Varid{f},\Varid{s})\mskip1.5mu]{}\<[E]%
\\
\>[B]{}\Varid{unaryOp}\;\anonymous \;\anonymous \;\anonymous \mathrel{=}[\mskip1.5mu \mskip1.5mu]{}\<[E]%
\\[\blanklineskip]%
\>[B]{}\Varid{negation}\mathbin{::}\Conid{Parser}{}\<[E]%
\\
\>[B]{}\Varid{negation}\mathrel{=}\Varid{unaryOp}\;\text{\ttfamily '\char126 '}\;\Varid{negate}{}\<[E]%
\\[\blanklineskip]%
\>[B]{}\Varid{factorial}\mathbin{::}\Conid{Parser}{}\<[E]%
\\
\>[B]{}\Varid{factorial}\mathrel{=}\Varid{unaryOp}\;\text{\ttfamily '!'}\;(\lambda \Varid{n}\to \Varid{product}\;[\mskip1.5mu \mathrm{1}\mathinner{\ldotp\ldotp}\Varid{n}\mskip1.5mu]){}\<[E]%
\ColumnHook
\end{hscode}\resethooks
\end{document}
