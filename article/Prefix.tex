\documentclass[a4paper,10pt]{article}
\usepackage{longtable,geometry}
\usepackage{tabularx}
\usepackage[english]{babel}
\usepackage[latin1]{inputenc}
\usepackage[babel]{csquotes}
\usepackage{multicol}
\usepackage{enumitem}
\usepackage{array}
\usepackage{fancyhdr}
\pagestyle{plain}
\geometry{dvips,a4paper,margin=1.5in}
\usepackage[usenames,dvipsnames]{xcolor}
\usepackage{amssymb,amsmath} 
\usepackage{afterpage}
\usepackage{pagecolor}
\usepackage{fancybox}
\usepackage{listings}
\usepackage{verbatim}
\usepackage{fancyvrb}
\usepackage{moreverb}
\usepackage{txfonts}
\usepackage{hyperref}
\usepackage{bytefield}
\usepackage{rotating}
\usepackage{tikz}
\usepackage[linguistics]{forest}
\usepackage{wallpaper}

%% ODER: format ==         = "\mathrel{==}"
%% ODER: format /=         = "\neq "
%
%
\makeatletter
\@ifundefined{lhs2tex.lhs2tex.sty.read}%
  {\@namedef{lhs2tex.lhs2tex.sty.read}{}%
   \newcommand\SkipToFmtEnd{}%
   \newcommand\EndFmtInput{}%
   \long\def\SkipToFmtEnd#1\EndFmtInput{}%
  }\SkipToFmtEnd

\newcommand\ReadOnlyOnce[1]{\@ifundefined{#1}{\@namedef{#1}{}}\SkipToFmtEnd}
\usepackage{amstext}
\usepackage{amssymb}
\usepackage{stmaryrd}
\DeclareFontFamily{OT1}{cmtex}{}
\DeclareFontShape{OT1}{cmtex}{m}{n}
  {<5><6><7><8>cmtex8
   <9>cmtex9
   <10><10.95><12><14.4><17.28><20.74><24.88>cmtex10}{}
\DeclareFontShape{OT1}{cmtex}{m}{it}
  {<-> ssub * cmtt/m/it}{}
\newcommand{\texfamily}{\fontfamily{cmtex}\selectfont}
\DeclareFontShape{OT1}{cmtt}{bx}{n}
  {<5><6><7><8>cmtt8
   <9>cmbtt9
   <10><10.95><12><14.4><17.28><20.74><24.88>cmbtt10}{}
\DeclareFontShape{OT1}{cmtex}{bx}{n}
  {<-> ssub * cmtt/bx/n}{}
\newcommand{\tex}[1]{\text{\texfamily#1}}	% NEU

\newcommand{\Sp}{\hskip.33334em\relax}


\newcommand{\Conid}[1]{\mathit{#1}}
\newcommand{\Varid}[1]{\mathit{#1}}
\newcommand{\anonymous}{\kern0.06em \vbox{\hrule\@width.5em}}
\newcommand{\plus}{\mathbin{+\!\!\!+}}
\newcommand{\bind}{\mathbin{>\!\!\!>\mkern-6.7mu=}}
\newcommand{\rbind}{\mathbin{=\mkern-6.7mu<\!\!\!<}}% suggested by Neil Mitchell
\newcommand{\sequ}{\mathbin{>\!\!\!>}}
\renewcommand{\leq}{\leqslant}
\renewcommand{\geq}{\geqslant}
\usepackage{polytable}

%mathindent has to be defined
\@ifundefined{mathindent}%
  {\newdimen\mathindent\mathindent\leftmargini}%
  {}%

\def\resethooks{%
  \global\let\SaveRestoreHook\empty
  \global\let\ColumnHook\empty}
\newcommand*{\savecolumns}[1][default]%
  {\g@addto@macro\SaveRestoreHook{\savecolumns[#1]}}
\newcommand*{\restorecolumns}[1][default]%
  {\g@addto@macro\SaveRestoreHook{\restorecolumns[#1]}}
\newcommand*{\aligncolumn}[2]%
  {\g@addto@macro\ColumnHook{\column{#1}{#2}}}

\resethooks

\newcommand{\onelinecommentchars}{\quad-{}- }
\newcommand{\commentbeginchars}{\enskip\{-}
\newcommand{\commentendchars}{-\}\enskip}

\newcommand{\visiblecomments}{%
  \let\onelinecomment=\onelinecommentchars
  \let\commentbegin=\commentbeginchars
  \let\commentend=\commentendchars}

\newcommand{\invisiblecomments}{%
  \let\onelinecomment=\empty
  \let\commentbegin=\empty
  \let\commentend=\empty}

\visiblecomments

\newlength{\blanklineskip}
\setlength{\blanklineskip}{0.66084ex}

\newcommand{\hsindent}[1]{\quad}% default is fixed indentation
\let\hspre\empty
\let\hspost\empty
\newcommand{\NB}{\textbf{NB}}
\newcommand{\Todo}[1]{$\langle$\textbf{To do:}~#1$\rangle$}

\EndFmtInput
\makeatother
%
%


%
%
%
%
%
% This package provides two environments suitable to take the place
% of hscode, called "plainhscode" and "arrayhscode". 
%
% The plain environment surrounds each code block by vertical space,
% and it uses \abovedisplayskip and \belowdisplayskip to get spacing
% similar to formulas. Note that if these dimensions are changed,
% the spacing around displayed math formulas changes as well.
% All code is indented using \leftskip.
%
% Changed 19.08.2004 to reflect changes in colorcode. Should work with
% CodeGroup.sty.
%
\ReadOnlyOnce{polycode.fmt}%
\makeatletter

\newcommand{\hsnewpar}[1]%
  {{\parskip=0pt\parindent=0pt\par\vskip #1\noindent}}

% can be used, for instance, to redefine the code size, by setting the
% command to \small or something alike
\newcommand{\hscodestyle}{}

% The command \sethscode can be used to switch the code formatting
% behaviour by mapping the hscode environment in the subst directive
% to a new LaTeX environment.

\newcommand{\sethscode}[1]%
  {\expandafter\let\expandafter\hscode\csname #1\endcsname
   \expandafter\let\expandafter\endhscode\csname end#1\endcsname}

% "compatibility" mode restores the non-polycode.fmt layout.

\newenvironment{compathscode}%
  {\par\noindent
   \advance\leftskip\mathindent
   \hscodestyle
   \let\\=\@normalcr
   \let\hspre\(\let\hspost\)%
   \pboxed}%
  {\endpboxed\)%
   \par\noindent
   \ignorespacesafterend}

\newcommand{\compaths}{\sethscode{compathscode}}

% "plain" mode is the proposed default.
% It should now work with \centering.
% This required some changes. The old version
% is still available for reference as oldplainhscode.

\newenvironment{plainhscode}%
  {\hsnewpar\abovedisplayskip
   \advance\leftskip\mathindent
   \hscodestyle
   \let\hspre\(\let\hspost\)%
   \pboxed}%
  {\endpboxed%
   \hsnewpar\belowdisplayskip
   \ignorespacesafterend}

\newenvironment{oldplainhscode}%
  {\hsnewpar\abovedisplayskip
   \advance\leftskip\mathindent
   \hscodestyle
   \let\\=\@normalcr
   \(\pboxed}%
  {\endpboxed\)%
   \hsnewpar\belowdisplayskip
   \ignorespacesafterend}

% Here, we make plainhscode the default environment.

\newcommand{\plainhs}{\sethscode{plainhscode}}
\newcommand{\oldplainhs}{\sethscode{oldplainhscode}}
\plainhs

% The arrayhscode is like plain, but makes use of polytable's
% parray environment which disallows page breaks in code blocks.

\newenvironment{arrayhscode}%
  {\hsnewpar\abovedisplayskip
   \advance\leftskip\mathindent
   \hscodestyle
   \let\\=\@normalcr
   \(\parray}%
  {\endparray\)%
   \hsnewpar\belowdisplayskip
   \ignorespacesafterend}

\newcommand{\arrayhs}{\sethscode{arrayhscode}}

% The mathhscode environment also makes use of polytable's parray 
% environment. It is supposed to be used only inside math mode 
% (I used it to typeset the type rules in my thesis).

\newenvironment{mathhscode}%
  {\parray}{\endparray}

\newcommand{\mathhs}{\sethscode{mathhscode}}

% texths is similar to mathhs, but works in text mode.

\newenvironment{texthscode}%
  {\(\parray}{\endparray\)}

\newcommand{\texths}{\sethscode{texthscode}}

% The framed environment places code in a framed box.

\def\codeframewidth{\arrayrulewidth}
\RequirePackage{calc}

\newenvironment{framedhscode}%
  {\parskip=\abovedisplayskip\par\noindent
   \hscodestyle
   \arrayrulewidth=\codeframewidth
   \tabular{@{}|p{\linewidth-2\arraycolsep-2\arrayrulewidth-2pt}|@{}}%
   \hline\framedhslinecorrect\\{-1.5ex}%
   \let\endoflinesave=\\
   \let\\=\@normalcr
   \(\pboxed}%
  {\endpboxed\)%
   \framedhslinecorrect\endoflinesave{.5ex}\hline
   \endtabular
   \parskip=\belowdisplayskip\par\noindent
   \ignorespacesafterend}

\newcommand{\framedhslinecorrect}[2]%
  {#1[#2]}

\newcommand{\framedhs}{\sethscode{framedhscode}}

% The inlinehscode environment is an experimental environment
% that can be used to typeset displayed code inline.

\newenvironment{inlinehscode}%
  {\(\def\column##1##2{}%
   \let\>\undefined\let\<\undefined\let\\\undefined
   \newcommand\>[1][]{}\newcommand\<[1][]{}\newcommand\\[1][]{}%
   \def\fromto##1##2##3{##3}%
   \def\nextline{}}{\) }%

\newcommand{\inlinehs}{\sethscode{inlinehscode}}

% The joincode environment is a separate environment that
% can be used to surround and thereby connect multiple code
% blocks.

\newenvironment{joincode}%
  {\let\orighscode=\hscode
   \let\origendhscode=\endhscode
   \def\endhscode{\def\hscode{\endgroup\def\@currenvir{hscode}\\}\begingroup}
   %\let\SaveRestoreHook=\empty
   %\let\ColumnHook=\empty
   %\let\resethooks=\empty
   \orighscode\def\hscode{\endgroup\def\@currenvir{hscode}}}%
  {\origendhscode
   \global\let\hscode=\orighscode
   \global\let\endhscode=\origendhscode}%

\makeatother
\EndFmtInput
%
\newcommand\tab[1][1cm]{\hspace*{#1}}
\newcommand\hl{\vspace{02mm}\hrule\vspace{02mm}}
\begin{document}
\setlength{\parindent}{0em}
\section*{Let's write a program...}
\begin{minipage}{6cm}
Let's write a program that reads expressions in prefix notation, and writes their value as an output.
\end{minipage}
\begin{minipage}{10cm}
\begin{hscode}\SaveRestoreHook
\column{B}{@{}>{\hspre}l<{\hspost}@{}}%
\column{E}{@{}>{\hspre}l<{\hspost}@{}}%
\>[B]{}\Varid{main}\mathrel{=}\Varid{interact}\mathbin{\$}\Varid{unlines}\mathbin{\circ}\Varid{map}\;\Varid{prefix}\mathbin{\circ}\Varid{lines}{}\<[E]%
\ColumnHook
\end{hscode}\resethooks
\end{minipage} 
\hl
\begin{minipage}{6cm}
Here are examples of such expressions. Note how the \text{\ttfamily \char45{}} symbol can be interpreted as the minus sign or the subtraction operator.\\
\end{minipage}
\begin{minipage}{10cm}
\begin{hscode}\SaveRestoreHook
\column{B}{@{}>{\hspre}l<{\hspost}@{}}%
\column{E}{@{}>{\hspre}l<{\hspost}@{}}%
\>[B]{}\Varid{example1}\mathrel{=}\text{\ttfamily \char34 *+42~17!5\char34}{}\<[E]%
\\
\>[B]{}\Varid{example2}\mathrel{=}\text{\ttfamily \char34 +4-3~+10-5\char34}{}\<[E]%
\ColumnHook
\end{hscode}\resethooks
\end{minipage}
\hl
\begin{minipage}{6cm}
To calculate the value of an expression such as: \\\begin{center}\framebox{$*\,+\,42\,17\,!5$}\\\end{center} we need to collect its \emph{tokens} ($*$,\,$+$,\,$42$,\,$17$,\,$!$,\,$5$) in a \emph{synctatic tree}. 
\end{minipage}
\begin{minipage}{10cm}
\begin{center}
\begin{forest}
    [$*$
        [$+$
            [$42$]
            [$17$]
        ]
        [$!$
            [$5$]
        ]
    ]
\end{forest}
\end{center}
\end{minipage}
\begin{minipage}{6cm}
Then the tree can be \emph{evaluated} according to three rules:\\
a node containing a number evaluates to this number,\\
a node containing an unary operator applies this operator to its first subtree,\\
a node containing a binary operator applies this operator to its first and second subtrees.\\
\end{minipage}
\begin{minipage}{10cm}
\begin{minipage}{20mm}
\begin{center}
\begin{forest}
    [$*$
        [$+$
            [$42$]
            [$17$]
        ]
        [$!$
            [$5$]
        ]
    ]
\end{forest}
\end{center}
\end{minipage}
$\Longrightarrow$
\begin{minipage}{20mm}
\begin{center}
\begin{forest}
    [$*$
        [$59$]
        [$!$
            [$5$]
        ]
    ]
\end{forest}
\end{center}
\end{minipage}
$\Longrightarrow$
\begin{minipage}{20mm}
\begin{center}
\begin{forest}
    [$*$
        [$59$]
        [{$120$}]
    ]
\end{forest}
\end{center}
\end{minipage}
\end{minipage}
\end{document}
$\Longrightarrow$
\begin{minipage}{3cm}
\begin{center}
\begin{forest}
    [$7080$]
\end{forest}
\end{center}
\end{minipage}\\
The main task consist thus in \emph{parsing} the String given in input, yielding a synctatic tree which, if correctly formed, can be evaluated.
\begin{hscode}\SaveRestoreHook
\column{B}{@{}>{\hspre}l<{\hspost}@{}}%
\column{5}{@{}>{\hspre}l<{\hspost}@{}}%
\column{16}{@{}>{\hspre}l<{\hspost}@{}}%
\column{E}{@{}>{\hspre}l<{\hspost}@{}}%
\>[B]{}\Varid{prefix}\mathbin{::}\Conid{String}\to \Conid{String}{}\<[E]%
\\
\>[B]{}\Varid{prefix}\;\Varid{s}\mathrel{=}\mathbf{case}\;\Varid{parse}\;\Varid{s}\;\mathbf{of}{}\<[E]%
\\
\>[B]{}\hsindent{5}{}\<[5]%
\>[5]{}[\mskip1.5mu (\Varid{tree},\anonymous )\mskip1.5mu]\to \Varid{show}\;(\Varid{eval}\;\Varid{tree}){}\<[E]%
\\
\>[B]{}\hsindent{5}{}\<[5]%
\>[5]{}[\mskip1.5mu \mskip1.5mu]{}\<[16]%
\>[16]{}\to \text{\ttfamily \char34 incorrect~prefix~expression\char34}{}\<[E]%
\ColumnHook
\end{hscode}\resethooks
\section*{Some data types}
We have to deal with values and functions of arity 1 and 2.
\begin{hscode}\SaveRestoreHook
\column{B}{@{}>{\hspre}l<{\hspost}@{}}%
\column{13}{@{}>{\hspre}l<{\hspost}@{}}%
\column{E}{@{}>{\hspre}l<{\hspost}@{}}%
\>[B]{}\mathbf{type}\;\Conid{Value}\mathrel{=}\Conid{Integer}{}\<[E]%
\\
\>[B]{}\mathbf{type}\;\Conid{Unary}{}\<[13]%
\>[13]{}\mathrel{=}\Conid{Value}\to \Conid{Value}{}\<[E]%
\\
\>[B]{}\mathbf{type}\;\Conid{Binary}\mathrel{=}\Conid{Value}\to \Conid{Value}\to \Conid{Value}{}\<[E]%
\ColumnHook
\end{hscode}\resethooks
A token can either be a numeric value, an unary operator or a binary operator.
\begin{hscode}\SaveRestoreHook
\column{B}{@{}>{\hspre}l<{\hspost}@{}}%
\column{12}{@{}>{\hspre}l<{\hspost}@{}}%
\column{E}{@{}>{\hspre}l<{\hspost}@{}}%
\>[B]{}\mathbf{data}\;\Conid{Token}\mathrel{=}\Conid{V}\;\Conid{Value}{}\<[E]%
\\
\>[B]{}\hsindent{12}{}\<[12]%
\>[12]{}\mid \Conid{U}\;\Conid{Unary}{}\<[E]%
\\
\>[B]{}\hsindent{12}{}\<[12]%
\>[12]{}\mid \Conid{B}\;\Conid{Binary}{}\<[E]%
\ColumnHook
\end{hscode}\resethooks
A tree can contain 0, 1 or 2 subtrees.
\begin{hscode}\SaveRestoreHook
\column{B}{@{}>{\hspre}l<{\hspost}@{}}%
\column{13}{@{}>{\hspre}l<{\hspost}@{}}%
\column{E}{@{}>{\hspre}l<{\hspost}@{}}%
\>[B]{}\mathbf{data}\;\Conid{Tree}\;\Varid{a}\mathrel{=}\Conid{Nil}{}\<[E]%
\\
\>[B]{}\hsindent{13}{}\<[13]%
\>[13]{}\mid \Conid{Node}\;\Varid{a}\;(\Conid{Tree}\;\Varid{a})\;(\Conid{Tree}\;\Varid{a}){}\<[E]%
\ColumnHook
\end{hscode}\resethooks
A parser of type $a$ is a function from $String$ to a list of couples $[(a,String)]$. An empty list means that parsing the string led to no result, and a list with several couples means that it led to several possible results.
\begin{hscode}\SaveRestoreHook
\column{B}{@{}>{\hspre}l<{\hspost}@{}}%
\column{E}{@{}>{\hspre}l<{\hspost}@{}}%
\>[B]{}\mathbf{type}\;\Conid{Parser}\;\Varid{a}\mathrel{=}\Conid{String}\to [\mskip1.5mu (\Varid{a},\Conid{String})\mskip1.5mu]{}\<[E]%
\ColumnHook
\end{hscode}\resethooks
\section*{Evaluating expressions}
When evaluating the tree for a prefix expression, we need to examine the token at the root of this tree, and use the subtrees depending on what sort of token it is.
\begin{hscode}\SaveRestoreHook
\column{B}{@{}>{\hspre}l<{\hspost}@{}}%
\column{30}{@{}>{\hspre}l<{\hspost}@{}}%
\column{E}{@{}>{\hspre}l<{\hspost}@{}}%
\>[B]{}\Varid{eval}\mathbin{::}\Conid{Tree}\;\Conid{Token}\to \Conid{Value}{}\<[E]%
\\
\>[B]{}\Varid{eval}\;(\Conid{Node}\;(\Conid{V}\;\Varid{n})\;\anonymous \;\anonymous )\mathrel{=}\Varid{n}{}\<[E]%
\\
\>[B]{}\Varid{eval}\;(\Conid{Node}\;(\Conid{U}\;\Varid{f})\;\Varid{operand}\;\anonymous ){}\<[30]%
\>[30]{}\mathrel{=}\Varid{f}\;(\Varid{eval}\;\Varid{operand}){}\<[E]%
\\
\>[B]{}\Varid{eval}\;(\Conid{Node}\;(\Conid{B}\;\Varid{f})\;\Varid{operand1}\;\Varid{operand2})\mathrel{=}\Varid{f}\;(\Varid{eval}\;\Varid{operand1})\;(\Varid{eval}\;\Varid{operand2}){}\<[E]%
\ColumnHook
\end{hscode}\resethooks
\section*{Parsing a prefix expression}
Let's first parse numbers, using the \text{\ttfamily reads} parser already present in Haskell prelude. We must avoid parsing negative numbers, because although our notation allows for the minus sign to indicate either a binary subtraction or a negative number, we want the symbol to be interpreted by our parser, not being consumed by \text{\ttfamily reads}.
\begin{hscode}\SaveRestoreHook
\column{B}{@{}>{\hspre}l<{\hspost}@{}}%
\column{5}{@{}>{\hspre}l<{\hspost}@{}}%
\column{13}{@{}>{\hspre}l<{\hspost}@{}}%
\column{E}{@{}>{\hspre}l<{\hspost}@{}}%
\>[B]{}\Varid{num}\mathbin{::}\Conid{Parser}\;\Conid{Token}{}\<[E]%
\\
\>[B]{}\Varid{num}\;\Varid{s}\mathrel{=}\mathbf{case}\;\Varid{reads}\;\Varid{s}\;\mathbf{of}{}\<[E]%
\\
\>[B]{}\hsindent{5}{}\<[5]%
\>[5]{}[\mskip1.5mu \mskip1.5mu]\to [\mskip1.5mu \mskip1.5mu]{}\<[E]%
\\
\>[B]{}\hsindent{5}{}\<[5]%
\>[5]{}[\mskip1.5mu (\Varid{n},\Varid{s})\mskip1.5mu]\mid \Varid{n}\geq \mathrm{0}\to [\mskip1.5mu (\Conid{V}\;\Varid{n},\Varid{s})\mskip1.5mu]{}\<[E]%
\\
\>[5]{}\hsindent{8}{}\<[13]%
\>[13]{}\mid \Varid{otherwise}\to [\mskip1.5mu \mskip1.5mu]{}\<[E]%
\ColumnHook
\end{hscode}\resethooks
Let's define some operators that our parsers will recognize.
\begin{hscode}\SaveRestoreHook
\column{B}{@{}>{\hspre}l<{\hspost}@{}}%
\column{E}{@{}>{\hspre}l<{\hspost}@{}}%
\>[B]{}[\mskip1.5mu \Varid{sAdd},\Varid{sSub},\Varid{sMul},\Varid{sDiv},\Varid{sMod},\Varid{sNeg},\Varid{sFac}\mskip1.5mu]\mathrel{=}\text{\ttfamily \char34 +-*/\%-!\char34}{}\<[E]%
\ColumnHook
\end{hscode}\resethooks
To parse an unary operator, we need to recognize one of the symbols for such operators:
\begin{hscode}\SaveRestoreHook
\column{B}{@{}>{\hspre}l<{\hspost}@{}}%
\column{E}{@{}>{\hspre}l<{\hspost}@{}}%
\>[B]{}\Varid{fact}\;\Varid{n}\mathrel{=}\Varid{product}\;[\mskip1.5mu \mathrm{1}\mathinner{\ldotp\ldotp}\Varid{n}\mskip1.5mu]{}\<[E]%
\\[\blanklineskip]%
\>[B]{}\Varid{unaryOp}\mathbin{::}\Conid{Parser}\;\Conid{Token}{}\<[E]%
\\
\>[B]{}\Varid{unaryOp}\;(\Varid{c}\mathbin{:}\Varid{s})\mid \Varid{c}\equiv \Varid{sNeg}\mathrel{=}[\mskip1.5mu (\Conid{U}\;\Varid{negate},\Varid{s})\mskip1.5mu]{}\<[E]%
\\
\>[B]{}\Varid{unaryOp}\;(\Varid{c}\mathbin{:}\Varid{s})\mid \Varid{c}\equiv \Varid{sFac}\mathrel{=}[\mskip1.5mu (\Conid{U}\;\Varid{fact},\Varid{s})\mskip1.5mu]{}\<[E]%
\\
\>[B]{}\Varid{unaryOp}\;\anonymous \mathrel{=}[\mskip1.5mu \mskip1.5mu]{}\<[E]%
\ColumnHook
\end{hscode}\resethooks
The same logic applies to binary operators:
\begin{hscode}\SaveRestoreHook
\column{B}{@{}>{\hspre}l<{\hspost}@{}}%
\column{E}{@{}>{\hspre}l<{\hspost}@{}}%
\>[B]{}\Varid{binaryOp}\mathbin{::}\Conid{Parser}\;\Conid{Token}{}\<[E]%
\\
\>[B]{}\Varid{binaryOp}\;(\Varid{c}\mathbin{:}\Varid{s})\mid \Varid{c}\equiv \Varid{sAdd}\mathrel{=}[\mskip1.5mu (\Conid{B}\;(\mathbin{+}),\Varid{s})\mskip1.5mu]{}\<[E]%
\\
\>[B]{}\Varid{binaryOp}\;(\Varid{c}\mathbin{:}\Varid{s})\mid \Varid{c}\equiv \Varid{sSub}\mathrel{=}[\mskip1.5mu (\Conid{B}\;(\mathbin{-}),\Varid{s})\mskip1.5mu]{}\<[E]%
\\
\>[B]{}\Varid{binaryOp}\;(\Varid{c}\mathbin{:}\Varid{s})\mid \Varid{c}\equiv \Varid{sMul}\mathrel{=}[\mskip1.5mu (\Conid{B}\;(\mathbin{*}),\Varid{s})\mskip1.5mu]{}\<[E]%
\\
\>[B]{}\Varid{binaryOp}\;(\Varid{c}\mathbin{:}\Varid{s})\mid \Varid{c}\equiv \Varid{sDiv}\mathrel{=}[\mskip1.5mu (\Conid{B}\;\Varid{div},\Varid{s})\mskip1.5mu]{}\<[E]%
\\
\>[B]{}\Varid{binaryOp}\;(\Varid{c}\mathbin{:}\Varid{s})\mid \Varid{c}\equiv \Varid{sMod}\mathrel{=}[\mskip1.5mu (\Conid{B}\;\Varid{mod},\Varid{s})\mskip1.5mu]{}\<[E]%
\\
\>[B]{}\Varid{binaryOp}\;\anonymous \mathrel{=}[\mskip1.5mu \mskip1.5mu]{}\<[E]%
\ColumnHook
\end{hscode}\resethooks
We know that expressions are formed with trees of tokens, so we need functions that will use our basic parsers and put their result into a tree.
\begin{hscode}\SaveRestoreHook
\column{B}{@{}>{\hspre}l<{\hspost}@{}}%
\column{5}{@{}>{\hspre}l<{\hspost}@{}}%
\column{E}{@{}>{\hspre}l<{\hspost}@{}}%
\>[B]{}\Varid{tree}\mathbin{::}\Conid{Parser}\;\Varid{a}\to \Conid{Parser}\;(\Conid{Tree}\;\Varid{a}){}\<[E]%
\\
\>[B]{}\Varid{tree}\;\Varid{p}\mathrel{=}\Varid{fmap}\;\Varid{node}\mathbin{\circ}\Varid{p}{}\<[E]%
\\
\>[B]{}\hsindent{5}{}\<[5]%
\>[5]{}\mathbf{where}{}\<[E]%
\\
\>[B]{}\hsindent{5}{}\<[5]%
\>[5]{}\Varid{node}\;(\Varid{a},\Varid{s})\mathrel{=}(\Conid{Node}\;\Varid{a}\;\Conid{Nil}\;\Conid{Nil},\Varid{s}){}\<[E]%
\ColumnHook
\end{hscode}\resethooks
Since spaces can separate numbers from operators, we need a function to augment our parsers so that they consume spaces before recognizing tokens.
\begin{hscode}\SaveRestoreHook
\column{B}{@{}>{\hspre}l<{\hspost}@{}}%
\column{E}{@{}>{\hspre}l<{\hspost}@{}}%
\>[B]{}\Varid{spaces}\mathbin{::}\Conid{Parser}\;\Varid{a}\to \Conid{Parser}\;\Varid{a}{}\<[E]%
\\
\>[B]{}\Varid{spaces}\;\Varid{p}\;(\text{\ttfamily '~'}\mathbin{:}\Varid{s})\mathrel{=}\Varid{spaces}\;\Varid{p}\;\Varid{s}{}\<[E]%
\\
\>[B]{}\Varid{spaces}\;\Varid{p}\;\Varid{s}\mathrel{=}\Varid{p}\;\Varid{s}{}\<[E]%
\ColumnHook
\end{hscode}\resethooks
We can now create parsers that produce trees of tokens:
\begin{hscode}\SaveRestoreHook
\column{B}{@{}>{\hspre}l<{\hspost}@{}}%
\column{8}{@{}>{\hspre}l<{\hspost}@{}}%
\column{E}{@{}>{\hspre}l<{\hspost}@{}}%
\>[B]{}\Varid{number},\Varid{unary},\Varid{binary}\mathbin{::}\Conid{Parser}\;(\Conid{Tree}\;\Conid{Token}){}\<[E]%
\\
\>[B]{}\Varid{number}\mathrel{=}\Varid{tree}\;(\Varid{spaces}\;\Varid{num}){}\<[E]%
\\
\>[B]{}\Varid{unary}{}\<[8]%
\>[8]{}\mathrel{=}\Varid{tree}\;(\Varid{spaces}\;\Varid{unaryOp}){}\<[E]%
\\
\>[B]{}\Varid{binary}\mathrel{=}\Varid{tree}\;(\Varid{spaces}\;\Varid{binaryOp}){}\<[E]%
\ColumnHook
\end{hscode}\resethooks

We should be able to recognize a sequence of different tokens, so let's write a parser that is the composition two parsers applied in sequence. The expression \text{\ttfamily p~\char60{}\char38{}\char62{}~q} denotes a parser that fails if $p$ fails or $q$ fails. In other case each result 
\begin{hscode}\SaveRestoreHook
\column{B}{@{}>{\hspre}l<{\hspost}@{}}%
\column{5}{@{}>{\hspre}l<{\hspost}@{}}%
\column{9}{@{}>{\hspre}l<{\hspost}@{}}%
\column{11}{@{}>{\hspre}l<{\hspost}@{}}%
\column{31}{@{}>{\hspre}l<{\hspost}@{}}%
\column{E}{@{}>{\hspre}l<{\hspost}@{}}%
\>[B]{}(\mathbin{<\&>})\mathbin{::}\Conid{Parser}\;(\Conid{Tree}\;\Varid{a})\to \Conid{Parser}\;(\Conid{Tree}\;\Varid{a})\to \Conid{Parser}\;(\Conid{Tree}\;\Varid{a}){}\<[E]%
\\
\>[B]{}(\Varid{parserA}\mathbin{<\&>}\Varid{parserB})\;\Varid{s}\mathrel{=}\mathbf{case}\;\Varid{parserA}\;\Varid{s}\;\mathbf{of}{}\<[E]%
\\
\>[B]{}\hsindent{5}{}\<[5]%
\>[5]{}[\mskip1.5mu \mskip1.5mu]\to [\mskip1.5mu \mskip1.5mu]{}\<[E]%
\\
\>[B]{}\hsindent{5}{}\<[5]%
\>[5]{}\Varid{rs}\to [\mskip1.5mu (\Varid{grow}\;\Varid{a}\;\Varid{b},\Varid{u}){}\<[E]%
\\
\>[5]{}\hsindent{6}{}\<[11]%
\>[11]{}\mid (\Varid{a},\Varid{t})\leftarrow \Varid{rs}{}\<[E]%
\\
\>[5]{}\hsindent{6}{}\<[11]%
\>[11]{},(\Varid{b},\Varid{u})\leftarrow \Varid{parserB}\;\Varid{t}\mskip1.5mu]{}\<[E]%
\\
\>[5]{}\hsindent{4}{}\<[9]%
\>[9]{}\mathbf{where}{}\<[E]%
\\
\>[5]{}\hsindent{4}{}\<[9]%
\>[9]{}\Varid{grow}\mathbin{::}\Conid{Tree}\;\Varid{a}\to \Conid{Tree}\;\Varid{a}\to \Conid{Tree}\;\Varid{a}{}\<[E]%
\\
\>[5]{}\hsindent{4}{}\<[9]%
\>[9]{}\Varid{grow}\;\Conid{Nil}\;\Varid{t}\mathrel{=}\Varid{t}{}\<[E]%
\\
\>[5]{}\hsindent{4}{}\<[9]%
\>[9]{}\Varid{grow}\;(\Conid{Node}\;\Varid{a}\;\Conid{Nil}\;\Conid{Nil})\;\Varid{t}\mathrel{=}\Conid{Node}\;\Varid{a}\;\Varid{t}\;\Conid{Nil}{}\<[E]%
\\
\>[5]{}\hsindent{4}{}\<[9]%
\>[9]{}\Varid{grow}\;(\Conid{Node}\;\Varid{a}\;\Varid{l}\;\Conid{Nil})\;{}\<[31]%
\>[31]{}\Varid{t}\mathrel{=}\Conid{Node}\;\Varid{a}\;\Varid{l}\;\Varid{t}{}\<[E]%
\ColumnHook
\end{hscode}\resethooks
We also want to combine two parsers so that one or the other succeeds, so let's write a parser that is defined by the alternative of two parsers.
\begin{hscode}\SaveRestoreHook
\column{B}{@{}>{\hspre}l<{\hspost}@{}}%
\column{5}{@{}>{\hspre}l<{\hspost}@{}}%
\column{E}{@{}>{\hspre}l<{\hspost}@{}}%
\>[B]{}\Varid{altP}\mathbin{::}\Conid{Parser}\;(\Conid{Tree}\;\Varid{a})\to \Conid{Parser}\;(\Conid{Tree}\;\Varid{a})\to \Conid{Parser}\;(\Conid{Tree}\;\Varid{a}){}\<[E]%
\\
\>[B]{}\Varid{altP}\;\Varid{parserA}\;\Varid{parserB}\;\Varid{s}\mathrel{=}\mathbf{case}\;\Varid{parserA}\;\Varid{s}\;\mathbf{of}{}\<[E]%
\\
\>[B]{}\hsindent{5}{}\<[5]%
\>[5]{}[\mskip1.5mu \mskip1.5mu]\to \Varid{parserB}\;\Varid{s}{}\<[E]%
\\
\>[B]{}\hsindent{5}{}\<[5]%
\>[5]{}\Varid{rs}\to \Varid{rs}{}\<[E]%
\ColumnHook
\end{hscode}\resethooks
Defining operators for these function will make the code more expressive.
\begin{hscode}\SaveRestoreHook
\column{B}{@{}>{\hspre}l<{\hspost}@{}}%
\column{E}{@{}>{\hspre}l<{\hspost}@{}}%
\>[B]{}\mathbf{infixl}\;\mathrm{2}\mathbin{<|>}{}\<[E]%
\\
\>[B]{}\mathbf{infixl}\;\mathrm{3}\mathbin{<\&>}{}\<[E]%
\\[\blanklineskip]%
\>[B]{}(\mathbin{<|>})\mathrel{=}\Varid{altP}{}\<[E]%
\ColumnHook
\end{hscode}\resethooks
Now we can define a parser for prefix expressions:
\begin{hscode}\SaveRestoreHook
\column{B}{@{}>{\hspre}l<{\hspost}@{}}%
\column{11}{@{}>{\hspre}l<{\hspost}@{}}%
\column{E}{@{}>{\hspre}l<{\hspost}@{}}%
\>[B]{}\Varid{expression}\mathbin{::}\Conid{Parser}\;(\Conid{Tree}\;\Conid{Token}){}\<[E]%
\\
\>[B]{}\Varid{expression}\mathrel{=}\Varid{number}{}\<[E]%
\\
\>[B]{}\hsindent{11}{}\<[11]%
\>[11]{}\mathbin{<|>}\Varid{unary}\mathbin{<\&>}\Varid{expression}{}\<[E]%
\\
\>[B]{}\hsindent{11}{}\<[11]%
\>[11]{}\mathbin{<|>}\Varid{binary}\mathbin{<\&>}\Varid{expression}\mathbin{<\&>}\Varid{expression}{}\<[E]%
\ColumnHook
\end{hscode}\resethooks

The \emph{parse} function that is called by \emph{prefix} is the same as the one that parse expressions:\\
\begin{hscode}\SaveRestoreHook
\column{B}{@{}>{\hspre}l<{\hspost}@{}}%
\column{E}{@{}>{\hspre}l<{\hspost}@{}}%
\>[B]{}\Varid{parse}\mathrel{=}\Varid{expression}{}\<[E]%
\ColumnHook
\end{hscode}\resethooks
\end{document}
