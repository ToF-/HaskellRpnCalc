\documentclass[a4paper,10pt]{article}
\usepackage{longtable,geometry}
\usepackage{tabularx}
\usepackage[english]{babel}
\usepackage[latin1]{inputenc}
\usepackage[babel]{csquotes}
\usepackage{multicol}
\usepackage{enumitem}
\usepackage{array}
\usepackage{fancyhdr}
\pagestyle{plain}
\geometry{dvips,a4paper,margin=1.5in}
\usepackage[usenames,dvipsnames]{xcolor}
\usepackage{amssymb,amsmath} 
\usepackage{afterpage}
\usepackage{pagecolor}
\usepackage{fancybox}
\usepackage{listings}
\usepackage{verbatim}
\usepackage{fancyvrb}
\usepackage{moreverb}
\usepackage{txfonts}
\usepackage{hyperref}
\usepackage{bytefield}
\usepackage{rotating}
\usepackage{tikz}
\usepackage[linguistics]{forest}
\usepackage{wallpaper}

%% ODER: format ==         = "\mathrel{==}"
%% ODER: format /=         = "\neq "
%
%
\makeatletter
\@ifundefined{lhs2tex.lhs2tex.sty.read}%
  {\@namedef{lhs2tex.lhs2tex.sty.read}{}%
   \newcommand\SkipToFmtEnd{}%
   \newcommand\EndFmtInput{}%
   \long\def\SkipToFmtEnd#1\EndFmtInput{}%
  }\SkipToFmtEnd

\newcommand\ReadOnlyOnce[1]{\@ifundefined{#1}{\@namedef{#1}{}}\SkipToFmtEnd}
\usepackage{amstext}
\usepackage{amssymb}
\usepackage{stmaryrd}
\DeclareFontFamily{OT1}{cmtex}{}
\DeclareFontShape{OT1}{cmtex}{m}{n}
  {<5><6><7><8>cmtex8
   <9>cmtex9
   <10><10.95><12><14.4><17.28><20.74><24.88>cmtex10}{}
\DeclareFontShape{OT1}{cmtex}{m}{it}
  {<-> ssub * cmtt/m/it}{}
\newcommand{\texfamily}{\fontfamily{cmtex}\selectfont}
\DeclareFontShape{OT1}{cmtt}{bx}{n}
  {<5><6><7><8>cmtt8
   <9>cmbtt9
   <10><10.95><12><14.4><17.28><20.74><24.88>cmbtt10}{}
\DeclareFontShape{OT1}{cmtex}{bx}{n}
  {<-> ssub * cmtt/bx/n}{}
\newcommand{\tex}[1]{\text{\texfamily#1}}	% NEU

\newcommand{\Sp}{\hskip.33334em\relax}


\newcommand{\Conid}[1]{\mathit{#1}}
\newcommand{\Varid}[1]{\mathit{#1}}
\newcommand{\anonymous}{\kern0.06em \vbox{\hrule\@width.5em}}
\newcommand{\plus}{\mathbin{+\!\!\!+}}
\newcommand{\bind}{\mathbin{>\!\!\!>\mkern-6.7mu=}}
\newcommand{\rbind}{\mathbin{=\mkern-6.7mu<\!\!\!<}}% suggested by Neil Mitchell
\newcommand{\sequ}{\mathbin{>\!\!\!>}}
\renewcommand{\leq}{\leqslant}
\renewcommand{\geq}{\geqslant}
\usepackage{polytable}

%mathindent has to be defined
\@ifundefined{mathindent}%
  {\newdimen\mathindent\mathindent\leftmargini}%
  {}%

\def\resethooks{%
  \global\let\SaveRestoreHook\empty
  \global\let\ColumnHook\empty}
\newcommand*{\savecolumns}[1][default]%
  {\g@addto@macro\SaveRestoreHook{\savecolumns[#1]}}
\newcommand*{\restorecolumns}[1][default]%
  {\g@addto@macro\SaveRestoreHook{\restorecolumns[#1]}}
\newcommand*{\aligncolumn}[2]%
  {\g@addto@macro\ColumnHook{\column{#1}{#2}}}

\resethooks

\newcommand{\onelinecommentchars}{\quad-{}- }
\newcommand{\commentbeginchars}{\enskip\{-}
\newcommand{\commentendchars}{-\}\enskip}

\newcommand{\visiblecomments}{%
  \let\onelinecomment=\onelinecommentchars
  \let\commentbegin=\commentbeginchars
  \let\commentend=\commentendchars}

\newcommand{\invisiblecomments}{%
  \let\onelinecomment=\empty
  \let\commentbegin=\empty
  \let\commentend=\empty}

\visiblecomments

\newlength{\blanklineskip}
\setlength{\blanklineskip}{0.66084ex}

\newcommand{\hsindent}[1]{\quad}% default is fixed indentation
\let\hspre\empty
\let\hspost\empty
\newcommand{\NB}{\textbf{NB}}
\newcommand{\Todo}[1]{$\langle$\textbf{To do:}~#1$\rangle$}

\EndFmtInput
\makeatother
%
%


%
%
%
%
%
% This package provides two environments suitable to take the place
% of hscode, called "plainhscode" and "arrayhscode". 
%
% The plain environment surrounds each code block by vertical space,
% and it uses \abovedisplayskip and \belowdisplayskip to get spacing
% similar to formulas. Note that if these dimensions are changed,
% the spacing around displayed math formulas changes as well.
% All code is indented using \leftskip.
%
% Changed 19.08.2004 to reflect changes in colorcode. Should work with
% CodeGroup.sty.
%
\ReadOnlyOnce{polycode.fmt}%
\makeatletter

\newcommand{\hsnewpar}[1]%
  {{\parskip=0pt\parindent=0pt\par\vskip #1\noindent}}

% can be used, for instance, to redefine the code size, by setting the
% command to \small or something alike
\newcommand{\hscodestyle}{}

% The command \sethscode can be used to switch the code formatting
% behaviour by mapping the hscode environment in the subst directive
% to a new LaTeX environment.

\newcommand{\sethscode}[1]%
  {\expandafter\let\expandafter\hscode\csname #1\endcsname
   \expandafter\let\expandafter\endhscode\csname end#1\endcsname}

% "compatibility" mode restores the non-polycode.fmt layout.

\newenvironment{compathscode}%
  {\par\noindent
   \advance\leftskip\mathindent
   \hscodestyle
   \let\\=\@normalcr
   \let\hspre\(\let\hspost\)%
   \pboxed}%
  {\endpboxed\)%
   \par\noindent
   \ignorespacesafterend}

\newcommand{\compaths}{\sethscode{compathscode}}

% "plain" mode is the proposed default.
% It should now work with \centering.
% This required some changes. The old version
% is still available for reference as oldplainhscode.

\newenvironment{plainhscode}%
  {\hsnewpar\abovedisplayskip
   \advance\leftskip\mathindent
   \hscodestyle
   \let\hspre\(\let\hspost\)%
   \pboxed}%
  {\endpboxed%
   \hsnewpar\belowdisplayskip
   \ignorespacesafterend}

\newenvironment{oldplainhscode}%
  {\hsnewpar\abovedisplayskip
   \advance\leftskip\mathindent
   \hscodestyle
   \let\\=\@normalcr
   \(\pboxed}%
  {\endpboxed\)%
   \hsnewpar\belowdisplayskip
   \ignorespacesafterend}

% Here, we make plainhscode the default environment.

\newcommand{\plainhs}{\sethscode{plainhscode}}
\newcommand{\oldplainhs}{\sethscode{oldplainhscode}}
\plainhs

% The arrayhscode is like plain, but makes use of polytable's
% parray environment which disallows page breaks in code blocks.

\newenvironment{arrayhscode}%
  {\hsnewpar\abovedisplayskip
   \advance\leftskip\mathindent
   \hscodestyle
   \let\\=\@normalcr
   \(\parray}%
  {\endparray\)%
   \hsnewpar\belowdisplayskip
   \ignorespacesafterend}

\newcommand{\arrayhs}{\sethscode{arrayhscode}}

% The mathhscode environment also makes use of polytable's parray 
% environment. It is supposed to be used only inside math mode 
% (I used it to typeset the type rules in my thesis).

\newenvironment{mathhscode}%
  {\parray}{\endparray}

\newcommand{\mathhs}{\sethscode{mathhscode}}

% texths is similar to mathhs, but works in text mode.

\newenvironment{texthscode}%
  {\(\parray}{\endparray\)}

\newcommand{\texths}{\sethscode{texthscode}}

% The framed environment places code in a framed box.

\def\codeframewidth{\arrayrulewidth}
\RequirePackage{calc}

\newenvironment{framedhscode}%
  {\parskip=\abovedisplayskip\par\noindent
   \hscodestyle
   \arrayrulewidth=\codeframewidth
   \tabular{@{}|p{\linewidth-2\arraycolsep-2\arrayrulewidth-2pt}|@{}}%
   \hline\framedhslinecorrect\\{-1.5ex}%
   \let\endoflinesave=\\
   \let\\=\@normalcr
   \(\pboxed}%
  {\endpboxed\)%
   \framedhslinecorrect\endoflinesave{.5ex}\hline
   \endtabular
   \parskip=\belowdisplayskip\par\noindent
   \ignorespacesafterend}

\newcommand{\framedhslinecorrect}[2]%
  {#1[#2]}

\newcommand{\framedhs}{\sethscode{framedhscode}}

% The inlinehscode environment is an experimental environment
% that can be used to typeset displayed code inline.

\newenvironment{inlinehscode}%
  {\(\def\column##1##2{}%
   \let\>\undefined\let\<\undefined\let\\\undefined
   \newcommand\>[1][]{}\newcommand\<[1][]{}\newcommand\\[1][]{}%
   \def\fromto##1##2##3{##3}%
   \def\nextline{}}{\) }%

\newcommand{\inlinehs}{\sethscode{inlinehscode}}

% The joincode environment is a separate environment that
% can be used to surround and thereby connect multiple code
% blocks.

\newenvironment{joincode}%
  {\let\orighscode=\hscode
   \let\origendhscode=\endhscode
   \def\endhscode{\def\hscode{\endgroup\def\@currenvir{hscode}\\}\begingroup}
   %\let\SaveRestoreHook=\empty
   %\let\ColumnHook=\empty
   %\let\resethooks=\empty
   \orighscode\def\hscode{\endgroup\def\@currenvir{hscode}}}%
  {\origendhscode
   \global\let\hscode=\orighscode
   \global\let\endhscode=\origendhscode}%

\makeatother
\EndFmtInput
%
\newcommand\tab[1][1cm]{\hspace*{#1}}
\begin{document}
\setlength{\parindent}{0em}
\section{Let's write a program...}
Let's write a program that reads expressions in prefix notation, and writes their value as an output. Here are examples of such expressions:\\

\begin{center}
\begin{tabular}{c c c}
\emph{prefix expression} & \emph{value} & \emph{infix equivalent}\\
\hline 
\hline
\text{\ttfamily \char42{}\char43{}42~17\char33{}5~} & $7080$ & $(42+17)\times!5$ \\
\text{\ttfamily \char43{}4\char45{}3~\char43{}10\char45{}5} & $2$ & $(4+3-(10+(-5)))$ \\
\hline
\end{tabular}\\
\end{center}
Note how the \text{\ttfamily \char45{}} symbol can be interpreted as the minus sign or the subtraction operator.

\begin{hscode}\SaveRestoreHook
\column{B}{@{}>{\hspre}l<{\hspost}@{}}%
\column{E}{@{}>{\hspre}l<{\hspost}@{}}%
\>[B]{}\mathbf{module}\;\Conid{Prefix}{}\<[E]%
\\
\>[B]{}\mathbf{where}{}\<[E]%
\ColumnHook
\end{hscode}\resethooks
Evaluating a prefix expression is easy because there is no surprise: the first element of any expression -- or sub expression -- always dictates how the rest of the expression should be interpreted. For instance in the expression $*+42\, 17\,!5$, from reading the $*$ symbol we know that we should find two operands in the rest of the expression. The first operand starts with the $+$ symbol, which indicates another binary operation, and so on. \\
Thus if we have determined each element of the expression and collected them into a list of \emph{tokens}, we can easily change that list into a tree, and then evaluate that tree. \\
\begin{center}
\begin{forest}
    [$*$
        [$+$
            [$42$]
            [$17$]
        ]
        [$!$
            [$5$]
        ]
    ]
\end{forest}
\end{center}
Thus if we can determine each element of the expression and collect them into a list of \emph{tokens}, evaluating such a sequence is straightforward. Recognizing each token in the expression, is a little more complicated and will be done by a parser function. \\

Let's start with the easy part:
\section{Evaluating a list of tokens}

A token in a prefix expression that has been correctly parsed can be:
\begin{itemize}
\item A number,
\item An operator for an unary function (e.g. factorial)
\item An operator for a binary function (e.g. multiplication)
\end{itemize}

\begin{hscode}\SaveRestoreHook
\column{B}{@{}>{\hspre}l<{\hspost}@{}}%
\column{12}{@{}>{\hspre}l<{\hspost}@{}}%
\column{E}{@{}>{\hspre}l<{\hspost}@{}}%
\>[B]{}\mathbf{type}\;\Conid{Number}\mathrel{=}\Conid{Integer}{}\<[E]%
\\
\>[B]{}\mathbf{data}\;\Conid{Token}\mathrel{=}\Conid{Num}\;\Conid{Number}{}\<[E]%
\\
\>[B]{}\hsindent{12}{}\<[12]%
\>[12]{}\mid \Conid{Op1}\;(\Conid{Number}\to \Conid{Number}){}\<[E]%
\\
\>[B]{}\hsindent{12}{}\<[12]%
\>[12]{}\mid \Conid{Op2}\;(\Conid{Number}\to \Conid{Number}\to \Conid{Number}){}\<[E]%
\ColumnHook
\end{hscode}\resethooks
For the sake of evaluation, expressions can be seen as list of tokens.
For example parsing the expression \text{\ttfamily \char42{}\char43{}42~17\char33{}5} should result in the following list:
\begin{hscode}\SaveRestoreHook
\column{B}{@{}>{\hspre}l<{\hspost}@{}}%
\column{E}{@{}>{\hspre}l<{\hspost}@{}}%
\>[B]{}\Varid{fact}\;\Varid{n}\mathrel{=}\Varid{product}\;[\mskip1.5mu \mathrm{1}\mathinner{\ldotp\ldotp}\Varid{n}\mskip1.5mu]{}\<[E]%
\\
\>[B]{}\Varid{example}\mathrel{=}[\mskip1.5mu \Conid{Op2}\;(\mathbin{*}),\Conid{Op2}\;(\mathbin{+}),\Conid{Num}\;\mathrm{42},\Conid{Num}\;\mathrm{17},\Conid{Op1}\;\Varid{fact},\Conid{Num}\;\mathrm{5}\mskip1.5mu]{}\<[E]%
\ColumnHook
\end{hscode}\resethooks
To evaluate a list of tokens representing a prefix expression, we need to examine the token at the head of the list. \\
\begin{hscode}\SaveRestoreHook
\column{B}{@{}>{\hspre}l<{\hspost}@{}}%
\column{E}{@{}>{\hspre}l<{\hspost}@{}}%
\>[B]{}\Varid{eval}\mathbin{::}[\mskip1.5mu \Conid{Token}\mskip1.5mu]\to (\Conid{Number},{}\<[E]%
\\
\>[B]{}\Conid{If}\;\Varid{this}\;\Varid{token}\;\Varid{matches}\;\Varid{the}\;\Varid{pattern}\lambda \Varid{verb}\mid \Conid{Num}\;\Varid{n}\mid ,\mathbf{then}\;\Varid{the}\;\Varid{value}\;\Varid{is}\mathbin{\$}\Varid{n}\mathbin{\$}\Varid{and}\;\Varid{the}\;\Varid{rest}\;\mathbf{of}\;\Varid{the}\;\Varid{list}\;\Varid{is}\;\Varid{to}\;\Varid{be}\;\Varid{evaluated}\;\Varid{further}\mathbin{.\char92 \char92 }{}\<[E]%
\\
\>[B]{}\Conid{If}\;\Varid{the}\;\Varid{head}\;\mathbf{of}\;\Varid{the}\;\Varid{list}\;\Varid{matches}\;\Varid{an}\;\Varid{unary}\;\Varid{operator},\lambda \Varid{verb}\mid \Conid{Op1}\;\Varid{f}\mid ,\Varid{we}\;\Varid{have}\;\Varid{to}\;\Varid{apply}\;\Varid{the}\;\Varid{function}\mathbin{\$}\Varid{f}\mathbin{\$}\Varid{to}\;\Varid{the}\;\Varid{value}\;\Varid{represented}\;\Varid{by}\;\Varid{the}\;\Varid{rest}\;\mathbf{of}\;\Varid{the}\;\Varid{list}\mathbin{\circ}\mathbin{\char92 \char92 }{}\<[E]%
\\
\>[B]{}\Conid{If}\;\Varid{the}\;\Varid{head}\;\mathbf{of}\;\Varid{the}\;\Varid{list}\;\Varid{matches}\;\Varid{a}\;\Varid{binary}\;\Varid{operator},\lambda \Varid{verb}\mid \Conid{Op2}\;\Varid{f}\mid ,\mathbf{then}\;\Varid{we}\;\Varid{have}\;\Varid{to}\;\Varid{first}\;\Varid{evaluate}\;\Varid{the}\;\Varid{rest}\;\mathbf{of}\;\Varid{the}\;\Varid{list},\Varid{which}\;\Varid{will}\;\Varid{give}\;\Varid{us}\;\Varid{the}\;\Varid{first}\;\Varid{operand}\;\Varid{value}\mathbin{\$}\Varid{n}\mathbin{\$}\Varid{and}\;\Varid{a}\;\Varid{remainging}\;\Varid{list},\Varid{and}\;\mathbf{then}\;\Varid{the}\;\Varid{evaluation}\;\Varid{amounts}\;\Varid{to}\;\Varid{evaluating}\;\Varid{a}\;\Varid{list}\;\Varid{starting}\;\Varid{with}\;\Varid{the}\;(\Varid{unary})\;\Varid{partial}\;\Varid{application}\mathbin{\$}\Varid{f}\;\Varid{n}\mathbin{\$}\Varid{to}\;\Varid{the}\;\Varid{value}\;\Varid{given}\;\Varid{by}\;\Varid{the}\;\Varid{rest}\;\mathbf{of}\;\Varid{the}\;\Varid{list}\mathbin{\circ}\mathbin{\char92 \char92 }{}\<[E]%
\\
\>[B]{}\Conid{Finally},\Varid{evaluating}\;\Varid{an}\;\Varid{empty}\;\Varid{list}\;\Varid{should}\;\Varid{never}\;\Varid{happen}\;(\Varid{since}\;\Varid{it}\;\Varid{means}\;\Varid{that}\;\Varid{the}\;\Varid{input}\;\Varid{couldn't}\;\Varid{be}\;\Varid{parsed}\;\mathbf{in}\;\Varid{a}\;\Varid{list}\;\mathbf{of}\;\Varid{tokens})\mathbin{\circ}{}\<[E]%
\\
\>[B]{}\lambda \Varid{begin}\;\{\mskip1.5mu \Varid{code}\mskip1.5mu\}{}\<[E]%
\\
\>[B]{}\Varid{eval}\mathbin{::}[\mskip1.5mu \Conid{Token}\mskip1.5mu]\to (\Conid{Number},[\mskip1.5mu \Conid{Token}\mskip1.5mu]){}\<[E]%
\\
\>[B]{}\Varid{eval}\;[\mskip1.5mu \mskip1.5mu]\mathrel{=}\Varid{error}\;\text{\ttfamily \char34 empty~token~list~given~to~eval\char34}{}\<[E]%
\\
\>[B]{}\Varid{eval}\;(\Conid{Num}\;\Varid{n}\mathbin{:}\Varid{ts})\mathrel{=}(\Varid{n},\Varid{ts}){}\<[E]%
\\
\>[B]{}\Varid{eval}\;(\Conid{Op1}\;\Varid{f}\mathbin{:}\Varid{ts})\mathrel{=}(\Varid{f}\;\Varid{n},\Varid{ts'})\;\mathbf{where}\;(\Varid{n},\Varid{ts'})\mathrel{=}\Varid{eval}\;\Varid{ts}{}\<[E]%
\\
\>[B]{}\Varid{eval}\;(\Conid{Op2}\;\Varid{f}\mathbin{:}\Varid{ts})\mathrel{=}\Varid{eval}\;(\Conid{Op1}\;(\Varid{f}\;\Varid{n})\mathbin{:}\Varid{ts'})\;\mathbf{where}\;(\Varid{n},\Varid{ts'})\mathrel{=}\Varid{eval}\;\Varid{ts}{}\<[E]%
\ColumnHook
\end{hscode}\resethooks
Thus the expression \text{\ttfamily fst~\char40{}eval~example\char41{}} should yield $7080$.\\
Here's how the expression \text{\ttfamily eval~\char91{}Op2~\char40{}\char43{}\char41{}\char44{}Num~42\char44{}~Num~17\char93{}} is evaluated:\\
\begin{tabbing}\ttfamily
~eval~\char91{}Op2~\char40{}\char43{}\char41{}\char44{}Num~42\char44{}~Num~17\char93{}\\
\ttfamily ~eval~\char40{}Op1~\char40{}\char40{}\char43{}\char41{}~n\char41{}\char58{}ts\char39{}\char41{}~where~\char40{}n\char44{}ts\char39{}\char41{}~\char61{}~eval~\char91{}Num~42\char44{}Num~17\char93{}\\
\ttfamily ~eval~\char40{}Op1~\char40{}\char40{}\char43{}\char41{}~n\char41{}\char58{}ts\char39{}\char41{}~where~\char40{}n\char44{}ts\char39{}\char41{}~\char61{}~\char40{}42\char44{}~\char91{}Num~17\char93{}\char41{}\\
\ttfamily ~eval~\char40{}Op1~\char40{}\char40{}\char43{}\char41{}~42\char41{}\char58{}\char91{}Num~17\char93{}\char41{}~\\
\ttfamily ~\char40{}\char40{}\char43{}\char41{}~42~n\char44{}ts\char39{}\char41{}~where~\char40{}n\char44{}ts\char39{}\char41{}~\char61{}~eval~\char91{}Num~17\char93{}\\
\ttfamily ~\char40{}\char40{}\char43{}\char41{}~42~n\char44{}ts\char39{}\char41{}~where~\char40{}n\char44{}ts\char39{}\char41{}~\char61{}~\char40{}17\char44{}\char91{}\char93{}\char41{}\\
\ttfamily ~\char40{}\char40{}\char43{}\char41{}~42~17\char44{}\char91{}\char93{}\char41{}~\\
\ttfamily ~\char40{}59\char44{}\char91{}\char93{}\char41{}
\end{tabbing}
\section{Parsing a prefix expression}
A parser is a function that scans a string and recognizes a token, or a given pattern. The result of the parsing is a list of tuples \text{\ttfamily \char40{}a\char44{}String\char41{}}, since there can be several distinct results from parsing a string.
\begin{hscode}\SaveRestoreHook
\column{B}{@{}>{\hspre}l<{\hspost}@{}}%
\column{E}{@{}>{\hspre}l<{\hspost}@{}}%
\>[B]{}\mathbf{type}\;\Conid{Parser}\;\Varid{a}\mathrel{=}\Conid{String}\to [\mskip1.5mu (\Varid{a},\Conid{String})\mskip1.5mu]{}\<[E]%
\ColumnHook
\end{hscode}\resethooks
Let's first parse numbers, using the \text{\ttfamily reads} parser already present in Haskell prelude. We must avoid parsing negative numbers, because in our notation, the minus sign denotes a subtraction, and we use the \text{\ttfamily \char126{}} operator to negate a number.
\begin{hscode}\SaveRestoreHook
\column{B}{@{}>{\hspre}l<{\hspost}@{}}%
\column{5}{@{}>{\hspre}l<{\hspost}@{}}%
\column{13}{@{}>{\hspre}l<{\hspost}@{}}%
\column{E}{@{}>{\hspre}l<{\hspost}@{}}%
\>[B]{}\Varid{num}\mathbin{::}\Conid{Parser}\;\Conid{Token}{}\<[E]%
\\
\>[B]{}\Varid{num}\;\Varid{s}\mathrel{=}\mathbf{case}\;\Varid{reads}\;\Varid{s}\;\mathbf{of}{}\<[E]%
\\
\>[B]{}\hsindent{5}{}\<[5]%
\>[5]{}[\mskip1.5mu \mskip1.5mu]\to [\mskip1.5mu \mskip1.5mu]{}\<[E]%
\\
\>[B]{}\hsindent{5}{}\<[5]%
\>[5]{}[\mskip1.5mu (\Varid{n},\Varid{s})\mskip1.5mu]\mid \Varid{n}\geq \mathrm{0}\to [\mskip1.5mu (\Conid{Num}\;\Varid{n},\Varid{s})\mskip1.5mu]{}\<[E]%
\\
\>[5]{}\hsindent{8}{}\<[13]%
\>[13]{}\mid \Varid{otherwise}\to [\mskip1.5mu \mskip1.5mu]{}\<[E]%
\ColumnHook
\end{hscode}\resethooks
Let's also define some operators that our parsers will recognize.
\begin{hscode}\SaveRestoreHook
\column{B}{@{}>{\hspre}l<{\hspost}@{}}%
\column{E}{@{}>{\hspre}l<{\hspost}@{}}%
\>[B]{}[\mskip1.5mu \Varid{sAdd},\Varid{sSub},\Varid{sMul},\Varid{sDiv},\Varid{sMod},\Varid{sNeg},\Varid{sFac}\mskip1.5mu]\mathrel{=}\text{\ttfamily \char34 +-*/\%-!\char34}{}\<[E]%
\ColumnHook
\end{hscode}\resethooks
To parse an unary operator, we need to recognize one of the symbols for such operators:
\begin{hscode}\SaveRestoreHook
\column{B}{@{}>{\hspre}l<{\hspost}@{}}%
\column{E}{@{}>{\hspre}l<{\hspost}@{}}%
\>[B]{}\Varid{unaryOp}\mathbin{::}\Conid{Parser}\;\Conid{Token}{}\<[E]%
\\
\>[B]{}\Varid{unaryOp}\;(\Varid{c}\mathbin{:}\Varid{s})\mid \Varid{c}\equiv \Varid{sNeg}\mathrel{=}[\mskip1.5mu (\Conid{Op1}\;\Varid{negate},\Varid{s})\mskip1.5mu]{}\<[E]%
\\
\>[B]{}\Varid{unaryOp}\;(\Varid{c}\mathbin{:}\Varid{s})\mid \Varid{c}\equiv \Varid{sFac}\mathrel{=}[\mskip1.5mu (\Conid{Op1}\;(\lambda \Varid{n}\to \Varid{product}\;[\mskip1.5mu \mathrm{1}\mathinner{\ldotp\ldotp}\Varid{n}\mskip1.5mu]),\Varid{s})\mskip1.5mu]{}\<[E]%
\\
\>[B]{}\Varid{unaryOp}\;\anonymous \mathrel{=}[\mskip1.5mu \mskip1.5mu]{}\<[E]%
\ColumnHook
\end{hscode}\resethooks
The same logic applies to binary operators:
\begin{hscode}\SaveRestoreHook
\column{B}{@{}>{\hspre}l<{\hspost}@{}}%
\column{E}{@{}>{\hspre}l<{\hspost}@{}}%
\>[B]{}\Varid{binaryOp}\mathbin{::}\Conid{Parser}\;\Conid{Token}{}\<[E]%
\\
\>[B]{}\Varid{binaryOp}\;(\Varid{c}\mathbin{:}\Varid{s})\mid \Varid{c}\equiv \Varid{sAdd}\mathrel{=}[\mskip1.5mu (\Conid{Op2}\;(\mathbin{+}),\Varid{s})\mskip1.5mu]{}\<[E]%
\\
\>[B]{}\Varid{binaryOp}\;(\Varid{c}\mathbin{:}\Varid{s})\mid \Varid{c}\equiv \Varid{sSub}\mathrel{=}[\mskip1.5mu (\Conid{Op2}\;(\mathbin{-}),\Varid{s})\mskip1.5mu]{}\<[E]%
\\
\>[B]{}\Varid{binaryOp}\;(\Varid{c}\mathbin{:}\Varid{s})\mid \Varid{c}\equiv \Varid{sMul}\mathrel{=}[\mskip1.5mu (\Conid{Op2}\;(\mathbin{*}),\Varid{s})\mskip1.5mu]{}\<[E]%
\\
\>[B]{}\Varid{binaryOp}\;(\Varid{c}\mathbin{:}\Varid{s})\mid \Varid{c}\equiv \Varid{sDiv}\mathrel{=}[\mskip1.5mu (\Conid{Op2}\;\Varid{div},\Varid{s})\mskip1.5mu]{}\<[E]%
\\
\>[B]{}\Varid{binaryOp}\;(\Varid{c}\mathbin{:}\Varid{s})\mid \Varid{c}\equiv \Varid{sMod}\mathrel{=}[\mskip1.5mu (\Conid{Op2}\;\Varid{mod},\Varid{s})\mskip1.5mu]{}\<[E]%
\\
\>[B]{}\Varid{binaryOp}\;\anonymous \mathrel{=}[\mskip1.5mu \mskip1.5mu]{}\<[E]%
\ColumnHook
\end{hscode}\resethooks
Since spaces can separate numbers from operators, we need a function to augment our parsers so that they consume spaces before recognizing tokens.
\begin{hscode}\SaveRestoreHook
\column{B}{@{}>{\hspre}l<{\hspost}@{}}%
\column{E}{@{}>{\hspre}l<{\hspost}@{}}%
\>[B]{}\Varid{spaces}\mathbin{::}\Conid{Parser}\;\Varid{a}\to \Conid{Parser}\;\Varid{a}{}\<[E]%
\\
\>[B]{}\Varid{spaces}\;\Varid{p}\;(\text{\ttfamily '~'}\mathbin{:}\Varid{s})\mathrel{=}\Varid{spaces}\;\Varid{p}\;\Varid{s}{}\<[E]%
\\
\>[B]{}\Varid{spaces}\;\Varid{p}\;\Varid{s}\mathrel{=}\Varid{p}\;\Varid{s}{}\<[E]%
\ColumnHook
\end{hscode}\resethooks
Since expressions are formed with lists of tokens, we need a function that will use our basic parsers and put their result into a list.
\begin{hscode}\SaveRestoreHook
\column{B}{@{}>{\hspre}l<{\hspost}@{}}%
\column{8}{@{}>{\hspre}l<{\hspost}@{}}%
\column{E}{@{}>{\hspre}l<{\hspost}@{}}%
\>[B]{}\Varid{list}\mathbin{::}\Conid{Parser}\;\Varid{a}\to \Conid{Parser}\;[\mskip1.5mu \Varid{a}\mskip1.5mu]{}\<[E]%
\\
\>[B]{}\Varid{list}\;\Varid{p}\mathrel{=}\Varid{map}\;(\lambda (\Varid{a},\Varid{s})\to ([\mskip1.5mu \Varid{a}\mskip1.5mu],\Varid{s}))\mathbin{\circ}\Varid{p}{}\<[E]%
\\[\blanklineskip]%
\>[B]{}\Varid{number}\mathrel{=}\Varid{list}\;(\Varid{spaces}\;\Varid{num}){}\<[E]%
\\
\>[B]{}\Varid{unary}{}\<[8]%
\>[8]{}\mathrel{=}\Varid{list}\;(\Varid{spaces}\;\Varid{unaryOp}){}\<[E]%
\\
\>[B]{}\Varid{binary}\mathrel{=}\Varid{list}\;(\Varid{spaces}\;\Varid{binaryOp}){}\<[E]%
\ColumnHook
\end{hscode}\resethooks

We should be able to recognize a sequence of different tokens, so let's write a parser that is defined by the sequence of two parsers:
\begin{hscode}\SaveRestoreHook
\column{B}{@{}>{\hspre}l<{\hspost}@{}}%
\column{5}{@{}>{\hspre}l<{\hspost}@{}}%
\column{11}{@{}>{\hspre}l<{\hspost}@{}}%
\column{E}{@{}>{\hspre}l<{\hspost}@{}}%
\>[B]{}\Varid{seqP}\mathbin{::}\Conid{Parser}\;[\mskip1.5mu \Varid{a}\mskip1.5mu]\to \Conid{Parser}\;[\mskip1.5mu \Varid{a}\mskip1.5mu]\to \Conid{Parser}\;[\mskip1.5mu \Varid{a}\mskip1.5mu]{}\<[E]%
\\
\>[B]{}\Varid{seqP}\;\Varid{parserA}\;\Varid{parserB}\;\Varid{s}\mathrel{=}\mathbf{case}\;\Varid{parserA}\;\Varid{s}\;\mathbf{of}{}\<[E]%
\\
\>[B]{}\hsindent{5}{}\<[5]%
\>[5]{}[\mskip1.5mu \mskip1.5mu]\to [\mskip1.5mu \mskip1.5mu]{}\<[E]%
\\
\>[B]{}\hsindent{5}{}\<[5]%
\>[5]{}\Varid{rs}\to [\mskip1.5mu (\Varid{a}\plus \Varid{b},\Varid{u}){}\<[E]%
\\
\>[5]{}\hsindent{6}{}\<[11]%
\>[11]{}\mid (\Varid{a},\Varid{t})\leftarrow \Varid{rs}{}\<[E]%
\\
\>[5]{}\hsindent{6}{}\<[11]%
\>[11]{},(\Varid{b},\Varid{u})\leftarrow \Varid{parserB}\;\Varid{t}\mskip1.5mu]{}\<[E]%
\ColumnHook
\end{hscode}\resethooks
We also want to combine two parsers so that one or the other succeeds, so let's write a parser that is defined by the alternative of two parsers.
\begin{hscode}\SaveRestoreHook
\column{B}{@{}>{\hspre}l<{\hspost}@{}}%
\column{5}{@{}>{\hspre}l<{\hspost}@{}}%
\column{E}{@{}>{\hspre}l<{\hspost}@{}}%
\>[B]{}\Varid{altP}\mathbin{::}\Conid{Parser}\;[\mskip1.5mu \Varid{a}\mskip1.5mu]\to \Conid{Parser}\;[\mskip1.5mu \Varid{a}\mskip1.5mu]\to \Conid{Parser}\;[\mskip1.5mu \Varid{a}\mskip1.5mu]{}\<[E]%
\\
\>[B]{}\Varid{altP}\;\Varid{parserA}\;\Varid{parserB}\;\Varid{s}\mathrel{=}\mathbf{case}\;\Varid{parserA}\;\Varid{s}\;\mathbf{of}{}\<[E]%
\\
\>[B]{}\hsindent{5}{}\<[5]%
\>[5]{}[\mskip1.5mu \mskip1.5mu]\to \Varid{parserB}\;\Varid{s}{}\<[E]%
\\
\>[B]{}\hsindent{5}{}\<[5]%
\>[5]{}\Varid{rs}\to \Varid{rs}{}\<[E]%
\ColumnHook
\end{hscode}\resethooks
Defining operators for these function will make the code more expressive.
\begin{hscode}\SaveRestoreHook
\column{B}{@{}>{\hspre}l<{\hspost}@{}}%
\column{E}{@{}>{\hspre}l<{\hspost}@{}}%
\>[B]{}\mathbf{infixl}\;\mathrm{2}\mathbin{<|>}{}\<[E]%
\\
\>[B]{}\mathbf{infixl}\;\mathrm{3}\mathbin{<.>}{}\<[E]%
\\[\blanklineskip]%
\>[B]{}(\mathbin{<|>})\mathrel{=}\Varid{altP}{}\<[E]%
\\
\>[B]{}(\mathbin{<.>})\mathrel{=}\Varid{seqP}{}\<[E]%
\ColumnHook
\end{hscode}\resethooks
Now we can define a parser for prefix expressions:
\begin{hscode}\SaveRestoreHook
\column{B}{@{}>{\hspre}l<{\hspost}@{}}%
\column{11}{@{}>{\hspre}l<{\hspost}@{}}%
\column{E}{@{}>{\hspre}l<{\hspost}@{}}%
\>[B]{}\Varid{expression}\mathbin{::}\Conid{Parser}\;[\mskip1.5mu \Conid{Token}\mskip1.5mu]{}\<[E]%
\\
\>[B]{}\Varid{expression}\mathrel{=}\Varid{number}{}\<[E]%
\\
\>[B]{}\hsindent{11}{}\<[11]%
\>[11]{}\mathbin{<|>}\Varid{binary}\mathbin{<.>}\Varid{expression}\mathbin{<.>}\Varid{expression}{}\<[E]%
\\
\>[B]{}\hsindent{11}{}\<[11]%
\>[11]{}\mathbin{<|>}\Varid{unary}\mathbin{<.>}\Varid{expression}{}\<[E]%
\ColumnHook
\end{hscode}\resethooks

Now we can define the main function to evaluate a prefix expression. If the expression has correctly been parsed, we evaluate the first list of tokens, and extract the number from the result.
\begin{hscode}\SaveRestoreHook
\column{B}{@{}>{\hspre}l<{\hspost}@{}}%
\column{5}{@{}>{\hspre}l<{\hspost}@{}}%
\column{E}{@{}>{\hspre}l<{\hspost}@{}}%
\>[B]{}\Varid{prefix}\mathbin{::}\Conid{String}\to \Conid{Number}{}\<[E]%
\\
\>[B]{}\Varid{prefix}\;\Varid{s}\mathrel{=}\mathbf{case}\;\Varid{expression}\;\Varid{s}\;\mathbf{of}{}\<[E]%
\\
\>[B]{}\hsindent{5}{}\<[5]%
\>[5]{}[\mskip1.5mu \mskip1.5mu]\to \Varid{error}\;\text{\ttfamily \char34 incorrect~prefix~expression\char34}{}\<[E]%
\\
\>[B]{}\hsindent{5}{}\<[5]%
\>[5]{}((\Varid{ts},\anonymous )\mathbin{:\char95 })\to \Varid{fst}\;(\Varid{eval}\;\Varid{ts}){}\<[E]%
\ColumnHook
\end{hscode}\resethooks
Our main program consist in evaluating prefix expressions:
\begin{hscode}\SaveRestoreHook
\column{B}{@{}>{\hspre}l<{\hspost}@{}}%
\column{E}{@{}>{\hspre}l<{\hspost}@{}}%
\>[B]{}\Varid{main}\mathrel{=}\Varid{interact}\mathbin{\$}\Varid{unlines}\mathbin{\circ}\Varid{map}\;(\Varid{show}\mathbin{\circ}\Varid{prefix})\mathbin{\circ}\Varid{lines}{}\<[E]%
\ColumnHook
\end{hscode}\resethooks

\end{document}
