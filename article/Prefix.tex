\documentclass[a4paper,10pt]{article}
\usepackage{longtable,geometry}
\usepackage[english]{babel}
\usepackage[latin1]{inputenc}
\usepackage[babel]{csquotes}
\usepackage{multicol}
\usepackage{enumitem}
\usepackage{array}
\usepackage{fancyhdr}
\pagestyle{plain}
\geometry{dvips,a4paper,margin=1.5in}
\usepackage[usenames,dvipsnames]{xcolor}
\usepackage{amssymb,amsmath} 
\usepackage{afterpage}
\usepackage{pagecolor}
\usepackage{fancybox}
\usepackage{listings}
\usepackage{verbatim}
\usepackage{fancyvrb}
\usepackage{moreverb}
\usepackage{txfonts}
\usepackage{hyperref}
\usepackage{bytefield}
\usepackage{rotating}
\usepackage{tikz}
\usepackage{wallpaper}

%% ODER: format ==         = "\mathrel{==}"
%% ODER: format /=         = "\neq "
%
%
\makeatletter
\@ifundefined{lhs2tex.lhs2tex.sty.read}%
  {\@namedef{lhs2tex.lhs2tex.sty.read}{}%
   \newcommand\SkipToFmtEnd{}%
   \newcommand\EndFmtInput{}%
   \long\def\SkipToFmtEnd#1\EndFmtInput{}%
  }\SkipToFmtEnd

\newcommand\ReadOnlyOnce[1]{\@ifundefined{#1}{\@namedef{#1}{}}\SkipToFmtEnd}
\usepackage{amstext}
\usepackage{amssymb}
\usepackage{stmaryrd}
\DeclareFontFamily{OT1}{cmtex}{}
\DeclareFontShape{OT1}{cmtex}{m}{n}
  {<5><6><7><8>cmtex8
   <9>cmtex9
   <10><10.95><12><14.4><17.28><20.74><24.88>cmtex10}{}
\DeclareFontShape{OT1}{cmtex}{m}{it}
  {<-> ssub * cmtt/m/it}{}
\newcommand{\texfamily}{\fontfamily{cmtex}\selectfont}
\DeclareFontShape{OT1}{cmtt}{bx}{n}
  {<5><6><7><8>cmtt8
   <9>cmbtt9
   <10><10.95><12><14.4><17.28><20.74><24.88>cmbtt10}{}
\DeclareFontShape{OT1}{cmtex}{bx}{n}
  {<-> ssub * cmtt/bx/n}{}
\newcommand{\tex}[1]{\text{\texfamily#1}}	% NEU

\newcommand{\Sp}{\hskip.33334em\relax}


\newcommand{\Conid}[1]{\mathit{#1}}
\newcommand{\Varid}[1]{\mathit{#1}}
\newcommand{\anonymous}{\kern0.06em \vbox{\hrule\@width.5em}}
\newcommand{\plus}{\mathbin{+\!\!\!+}}
\newcommand{\bind}{\mathbin{>\!\!\!>\mkern-6.7mu=}}
\newcommand{\rbind}{\mathbin{=\mkern-6.7mu<\!\!\!<}}% suggested by Neil Mitchell
\newcommand{\sequ}{\mathbin{>\!\!\!>}}
\renewcommand{\leq}{\leqslant}
\renewcommand{\geq}{\geqslant}
\usepackage{polytable}

%mathindent has to be defined
\@ifundefined{mathindent}%
  {\newdimen\mathindent\mathindent\leftmargini}%
  {}%

\def\resethooks{%
  \global\let\SaveRestoreHook\empty
  \global\let\ColumnHook\empty}
\newcommand*{\savecolumns}[1][default]%
  {\g@addto@macro\SaveRestoreHook{\savecolumns[#1]}}
\newcommand*{\restorecolumns}[1][default]%
  {\g@addto@macro\SaveRestoreHook{\restorecolumns[#1]}}
\newcommand*{\aligncolumn}[2]%
  {\g@addto@macro\ColumnHook{\column{#1}{#2}}}

\resethooks

\newcommand{\onelinecommentchars}{\quad-{}- }
\newcommand{\commentbeginchars}{\enskip\{-}
\newcommand{\commentendchars}{-\}\enskip}

\newcommand{\visiblecomments}{%
  \let\onelinecomment=\onelinecommentchars
  \let\commentbegin=\commentbeginchars
  \let\commentend=\commentendchars}

\newcommand{\invisiblecomments}{%
  \let\onelinecomment=\empty
  \let\commentbegin=\empty
  \let\commentend=\empty}

\visiblecomments

\newlength{\blanklineskip}
\setlength{\blanklineskip}{0.66084ex}

\newcommand{\hsindent}[1]{\quad}% default is fixed indentation
\let\hspre\empty
\let\hspost\empty
\newcommand{\NB}{\textbf{NB}}
\newcommand{\Todo}[1]{$\langle$\textbf{To do:}~#1$\rangle$}

\EndFmtInput
\makeatother
%
%


%
%
%
%
%
% This package provides two environments suitable to take the place
% of hscode, called "plainhscode" and "arrayhscode". 
%
% The plain environment surrounds each code block by vertical space,
% and it uses \abovedisplayskip and \belowdisplayskip to get spacing
% similar to formulas. Note that if these dimensions are changed,
% the spacing around displayed math formulas changes as well.
% All code is indented using \leftskip.
%
% Changed 19.08.2004 to reflect changes in colorcode. Should work with
% CodeGroup.sty.
%
\ReadOnlyOnce{polycode.fmt}%
\makeatletter

\newcommand{\hsnewpar}[1]%
  {{\parskip=0pt\parindent=0pt\par\vskip #1\noindent}}

% can be used, for instance, to redefine the code size, by setting the
% command to \small or something alike
\newcommand{\hscodestyle}{}

% The command \sethscode can be used to switch the code formatting
% behaviour by mapping the hscode environment in the subst directive
% to a new LaTeX environment.

\newcommand{\sethscode}[1]%
  {\expandafter\let\expandafter\hscode\csname #1\endcsname
   \expandafter\let\expandafter\endhscode\csname end#1\endcsname}

% "compatibility" mode restores the non-polycode.fmt layout.

\newenvironment{compathscode}%
  {\par\noindent
   \advance\leftskip\mathindent
   \hscodestyle
   \let\\=\@normalcr
   \let\hspre\(\let\hspost\)%
   \pboxed}%
  {\endpboxed\)%
   \par\noindent
   \ignorespacesafterend}

\newcommand{\compaths}{\sethscode{compathscode}}

% "plain" mode is the proposed default.
% It should now work with \centering.
% This required some changes. The old version
% is still available for reference as oldplainhscode.

\newenvironment{plainhscode}%
  {\hsnewpar\abovedisplayskip
   \advance\leftskip\mathindent
   \hscodestyle
   \let\hspre\(\let\hspost\)%
   \pboxed}%
  {\endpboxed%
   \hsnewpar\belowdisplayskip
   \ignorespacesafterend}

\newenvironment{oldplainhscode}%
  {\hsnewpar\abovedisplayskip
   \advance\leftskip\mathindent
   \hscodestyle
   \let\\=\@normalcr
   \(\pboxed}%
  {\endpboxed\)%
   \hsnewpar\belowdisplayskip
   \ignorespacesafterend}

% Here, we make plainhscode the default environment.

\newcommand{\plainhs}{\sethscode{plainhscode}}
\newcommand{\oldplainhs}{\sethscode{oldplainhscode}}
\plainhs

% The arrayhscode is like plain, but makes use of polytable's
% parray environment which disallows page breaks in code blocks.

\newenvironment{arrayhscode}%
  {\hsnewpar\abovedisplayskip
   \advance\leftskip\mathindent
   \hscodestyle
   \let\\=\@normalcr
   \(\parray}%
  {\endparray\)%
   \hsnewpar\belowdisplayskip
   \ignorespacesafterend}

\newcommand{\arrayhs}{\sethscode{arrayhscode}}

% The mathhscode environment also makes use of polytable's parray 
% environment. It is supposed to be used only inside math mode 
% (I used it to typeset the type rules in my thesis).

\newenvironment{mathhscode}%
  {\parray}{\endparray}

\newcommand{\mathhs}{\sethscode{mathhscode}}

% texths is similar to mathhs, but works in text mode.

\newenvironment{texthscode}%
  {\(\parray}{\endparray\)}

\newcommand{\texths}{\sethscode{texthscode}}

% The framed environment places code in a framed box.

\def\codeframewidth{\arrayrulewidth}
\RequirePackage{calc}

\newenvironment{framedhscode}%
  {\parskip=\abovedisplayskip\par\noindent
   \hscodestyle
   \arrayrulewidth=\codeframewidth
   \tabular{@{}|p{\linewidth-2\arraycolsep-2\arrayrulewidth-2pt}|@{}}%
   \hline\framedhslinecorrect\\{-1.5ex}%
   \let\endoflinesave=\\
   \let\\=\@normalcr
   \(\pboxed}%
  {\endpboxed\)%
   \framedhslinecorrect\endoflinesave{.5ex}\hline
   \endtabular
   \parskip=\belowdisplayskip\par\noindent
   \ignorespacesafterend}

\newcommand{\framedhslinecorrect}[2]%
  {#1[#2]}

\newcommand{\framedhs}{\sethscode{framedhscode}}

% The inlinehscode environment is an experimental environment
% that can be used to typeset displayed code inline.

\newenvironment{inlinehscode}%
  {\(\def\column##1##2{}%
   \let\>\undefined\let\<\undefined\let\\\undefined
   \newcommand\>[1][]{}\newcommand\<[1][]{}\newcommand\\[1][]{}%
   \def\fromto##1##2##3{##3}%
   \def\nextline{}}{\) }%

\newcommand{\inlinehs}{\sethscode{inlinehscode}}

% The joincode environment is a separate environment that
% can be used to surround and thereby connect multiple code
% blocks.

\newenvironment{joincode}%
  {\let\orighscode=\hscode
   \let\origendhscode=\endhscode
   \def\endhscode{\def\hscode{\endgroup\def\@currenvir{hscode}\\}\begingroup}
   %\let\SaveRestoreHook=\empty
   %\let\ColumnHook=\empty
   %\let\resethooks=\empty
   \orighscode\def\hscode{\endgroup\def\@currenvir{hscode}}}%
  {\origendhscode
   \global\let\hscode=\orighscode
   \global\let\endhscode=\origendhscode}%

\makeatother
\EndFmtInput
%
\newcommand\tab[1][1cm]{\hspace*{#1}}
\begin{document}
\setlength{\parindent}{0em}
Let's write a parser for expressions in prefix notation. Here are examples:\\

\text{\ttfamily \char42{}\char43{}42~17\char33{}5~} \tab $\rightarrow \tab (42+17)\times!5$ \\
\text{\ttfamily \char43{}4\char45{}3~\char43{}10\char126{}5} \tab $\rightarrow \tab (4+3-(10+(-5)))$ \\

We will need built-in functions to detect a space or a digit character, so let's import these. 
\begin{hscode}\SaveRestoreHook
\column{B}{@{}>{\hspre}l<{\hspost}@{}}%
\column{E}{@{}>{\hspre}l<{\hspost}@{}}%
\>[B]{}\mathbf{module}\;\Conid{Prefix}{}\<[E]%
\\
\>[B]{}\mathbf{where}{}\<[E]%
\\
\>[B]{}\mathbf{import}\;\Conid{\Conid{Data}.Char}\;(\Varid{isSpace},\Varid{isDigit},\Varid{digitToInt}){}\<[E]%
\ColumnHook
\end{hscode}\resethooks

Evaluating a prefix expression requires two steps:
\begin{itemize}
\item parsing the expression into \emph{tokens},
\item evaluating these tokens according to the rules of the prefix notation.
\end{itemize}

\section{Tokens, and how to evaluate them}

Let's define the possible tokens for our prefix notation. A token can be:
\begin{itemize}
\item A number,
\item An operator representing an unary function (e.g. factorial)
\item An operator representing a binary function (e.g. multiplication)
\end{itemize}

\begin{hscode}\SaveRestoreHook
\column{B}{@{}>{\hspre}l<{\hspost}@{}}%
\column{12}{@{}>{\hspre}l<{\hspost}@{}}%
\column{E}{@{}>{\hspre}l<{\hspost}@{}}%
\>[B]{}\mathbf{type}\;\Conid{Number}\mathrel{=}\Conid{Integer}{}\<[E]%
\\
\>[B]{}\mathbf{data}\;\Conid{Token}\mathrel{=}\Conid{Num}\;\Conid{Number}{}\<[E]%
\\
\>[B]{}\hsindent{12}{}\<[12]%
\>[12]{}\mid \Conid{Op1}\;(\Conid{Number}\to \Conid{Number}){}\<[E]%
\\
\>[B]{}\hsindent{12}{}\<[12]%
\>[12]{}\mid \Conid{Op2}\;(\Conid{Number}\to \Conid{Number}\to \Conid{Number}){}\<[E]%
\ColumnHook
\end{hscode}\resethooks
Since an unary operator should be followed by another expression, and a binary operator by two expressions, it is natural to represent a parsed expressionas as a list of tokens.
For example parsing the expression \text{\ttfamily \char42{}\char43{}42~17\char33{}5} should result in the following list:
\begin{hscode}\SaveRestoreHook
\column{B}{@{}>{\hspre}l<{\hspost}@{}}%
\column{E}{@{}>{\hspre}l<{\hspost}@{}}%
\>[B]{}\Varid{example}\mathrel{=}[\mskip1.5mu \Conid{Op2}\;(\mathbin{*}),\Conid{Op2}\;(\mathbin{+}),\Conid{Num}\;\mathrm{42},\Conid{Num}\;\mathrm{17},\Conid{Op1}\;(\lambda \Varid{n}\to \Varid{product}\;[\mskip1.5mu \mathrm{1}\mathinner{\ldotp\ldotp}\Varid{n}\mskip1.5mu]),\Conid{Num}\;\mathrm{5}\mskip1.5mu]{}\<[E]%
\ColumnHook
\end{hscode}\resethooks
To evaluate a list of tokens representing a prefix expression, we need to examine the token at the head of the list. If this token matches the pattern \text{\ttfamily Num~n}, then the value is $n$ and the rest of the list is to be evaluated further.\\
If the head of the list matches an unary operator, \text{\ttfamily Op1~f}, we have to apply the function $f$ to the value represented by the rest of the list. \\
If the head of the list matches a binary operator, \text{\ttfamily Op2~f}, then we have to first evaluate the rest of the list, which will give us the first operand value $n$ and a remainging list, and then the evaluation amounts to evaluating a list starting with the (unary) partial application $f n$ to the value given by the rest of the list. \\
Finally, evaluating an empty list should yield the (arbitrary) value $0$, and an empty list.
\begin{hscode}\SaveRestoreHook
\column{B}{@{}>{\hspre}l<{\hspost}@{}}%
\column{E}{@{}>{\hspre}l<{\hspost}@{}}%
\>[B]{}\Varid{eval}\mathbin{::}[\mskip1.5mu \Conid{Token}\mskip1.5mu]\to (\Conid{Number},[\mskip1.5mu \Conid{Token}\mskip1.5mu]){}\<[E]%
\\
\>[B]{}\Varid{eval}\;[\mskip1.5mu \mskip1.5mu]\mathrel{=}(\mathrm{0},[\mskip1.5mu \mskip1.5mu]){}\<[E]%
\\
\>[B]{}\Varid{eval}\;(\Conid{Num}\;\Varid{n}\mathbin{:}\Varid{ts})\mathrel{=}(\Varid{n},\Varid{ts}){}\<[E]%
\\
\>[B]{}\Varid{eval}\;(\Conid{Op1}\;\Varid{f}\mathbin{:}\Varid{ts})\mathrel{=}(\Varid{f}\;\Varid{n},\Varid{ts'})\;\mathbf{where}\;(\Varid{n},\Varid{ts'})\mathrel{=}\Varid{eval}\;\Varid{ts}{}\<[E]%
\\
\>[B]{}\Varid{eval}\;(\Conid{Op2}\;\Varid{f}\mathbin{:}\Varid{ts})\mathrel{=}\Varid{eval}\;(\Conid{Op1}\;(\Varid{f}\;\Varid{n})\mathbin{:}\Varid{ts'})\;\mathbf{where}\;(\Varid{n},\Varid{ts'})\mathrel{=}\Varid{eval}\;\Varid{ts}{}\<[E]%
\ColumnHook
\end{hscode}\resethooks
Thus the expression \text{\ttfamily fst~\char40{}eval~example\char41{}} should yield $7080$.\\
Here's how the expression \text{\ttfamily eval~\char91{}Op2~\char40{}\char43{}\char41{}\char44{}Num~42\char44{}~Num~17\char93{}} is evaluated:\\
\begin{tabbing}\ttfamily
~eval~\char91{}Op2~\char40{}\char43{}\char41{}\char44{}Num~42\char44{}~Num~17\char93{}\\
\ttfamily ~eval~\char40{}Op1~\char40{}\char40{}\char43{}\char41{}~n\char41{}\char58{}ts\char39{}\char41{}~where~\char40{}n\char44{}ts\char39{}\char41{}~\char61{}~eval~\char91{}Num~42\char44{}Num~17\char93{}\\
\ttfamily ~eval~\char40{}Op1~\char40{}\char40{}\char43{}\char41{}~n\char41{}\char58{}ts\char39{}\char41{}~where~\char40{}n\char44{}ts\char39{}\char41{}~\char61{}~\char40{}42\char44{}~\char91{}Num~17\char93{}\char41{}\\
\ttfamily ~eval~\char40{}Op1~\char40{}\char40{}\char43{}\char41{}~42\char41{}\char58{}\char91{}Num~17\char93{}\char41{}~\\
\ttfamily ~\char40{}\char40{}\char43{}\char41{}~42~n\char44{}ts\char39{}\char41{}~where~\char40{}n\char44{}ts\char39{}\char41{}~\char61{}~eval~\char91{}Num~17\char93{}\\
\ttfamily ~\char40{}\char40{}\char43{}\char41{}~42~n\char44{}ts\char39{}\char41{}~where~\char40{}n\char44{}ts\char39{}\char41{}~\char61{}~\char40{}17\char44{}\char91{}\char93{}\char41{}\\
\ttfamily ~\char40{}\char40{}\char43{}\char41{}~42~17\char44{}\char91{}\char93{}\char41{}~\\
\ttfamily ~\char40{}59\char44{}\char91{}\char93{}\char41{}
\end{tabbing}
\section{Parsing a prefix expression}
A parser is a function that scans a string and recognizes a token, or a given pattern. Since there can be several possibilities, (for example, the sequence starting with $+42-17\dots$ could denote the addition of $42$ and $(-17)$ followed by something we can't recognize yet, or the expression $42+(17-\dots)$), the result should consist in a list of possible result, each result including the token found and the part of the string that remains to be parsed.
\begin{hscode}\SaveRestoreHook
\column{B}{@{}>{\hspre}l<{\hspost}@{}}%
\column{E}{@{}>{\hspre}l<{\hspost}@{}}%
\>[B]{}\mathbf{type}\;\Conid{Parser}\;\Varid{a}\mathrel{=}\Conid{String}\to [\mskip1.5mu (\Varid{a},\Conid{String})\mskip1.5mu]{}\<[E]%
\ColumnHook
\end{hscode}\resethooks
Let's also define some operators that our parser will recognize.
\begin{hscode}\SaveRestoreHook
\column{B}{@{}>{\hspre}l<{\hspost}@{}}%
\column{E}{@{}>{\hspre}l<{\hspost}@{}}%
\>[B]{}[\mskip1.5mu \Varid{sAdd},\Varid{sSub},\Varid{sMul},\Varid{sDiv},\Varid{sMod},\Varid{sNeg},\Varid{sFac}\mskip1.5mu]\mathrel{=}\text{\ttfamily \char34 +-*/\%\char126 !\char34}{}\<[E]%
\ColumnHook
\end{hscode}\resethooks
\end{document}
